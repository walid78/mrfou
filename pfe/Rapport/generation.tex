\section{Génération de code à partir des noeuds altarica élémentaires.}

Etant donné que notre modèle altarica représente les comportements des éléments du système, nous allons ici faire le lien entre chaque comportement adopté par une entité et des portions de code implémentant ces comportements.

\subsection{Motor}

\begin{verbatim}
event
    	back		OnFwd(Out_X, Y); pour le moteur X, et Y choisi arbitrairement entre 1 et 100.
    	stop		OnFwd(Out_X, 0);
    	forward	OnRev(Out_X, Y); pour le moteur X, et Y choisi arbitrairement entre 1 et 100.

\end{verbatim}

La conjonction des évenements des 2 moteurs permettent de déplacer le robot de manière cohérente. Ceci correspond au noeud Moving.

\subsection{Moving}

\begin{verbatim}
event
	stop	OnFwd(Out_AC, 0);
	left	OnFwd(Out_C, 20); et OnRev(Out_A, 20); ou bien RotateMotorEx(OUT_AC, 40, 180, -100, true, true); (plus simple)
	forward	OnFwd(Out_AC, 40);
	right		OnFwd(Out_A, 20); et OnRev(Out_C, 20); ou bien RotateMotorEx(OUT_AC, 40, 180, 100, true, true); (plus simple)
	halfturn	RotateMotorEx(OUT_AC, 40, 360, 100, true, true);
\end{verbatim}

\subsection{LightSensor}

\begin{verbatim}
flow
    value : [0,2];	value = SensorUs(3); 3 étant dans notre cas le numéro du port sur lequel est branché le capteur lumineux.
\end{verbatim}

\subsection{UltraSonicSensor}

\begin{verbatim}
flow
    d : bool;	value = SensorUs(4); 4 étant dans notre cas le numéro du port sur lequel est branché le capteur ultrason.

\end{verbatim}

\subsection{BlueTooth}