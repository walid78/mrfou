  \subsection{Question 2}
  Le but ici est de faire la comparaison entre l'algorithme du projet vu
  dans la question précédente (\ref{part2q1} page \pageref{part2q1}) et
  l'alogorithme vu en cours basé sur un couplage maximum.\newline
  \indent L'algoritme du couplage maximum est implémenté dans le fichier
  \emph{Graph.cpp} par la méthode \emph{coverCourses()} et l'algorithme
  du projet est implémenté dans \emph{Tree.cpp} par la méthode
  \emph{coverProject()} et la fonction \emph{coverProject\_aux()}.\newline
  \indent Pour comparer ces deux algorithmes, nous allons comparer leur
  temps d'exécution et surtout la taille de la couverture trouvé. Le
  résultat de ces comparaisons se trouve dans le tableau \ref{tableau}
  page \pageref{tableau}. Toutes les comparaison sont faites sur trois
  graphes: \emph{ex32-12.gin, testTree1000.gin, testTree100.gin}.\newline
  \indent La comparaison des temps d'exécution ce fait à l'aide des
  fichiers \emph{timing.cc} et \emph{timing.h} que nous avons récuperés
  des projets \emph{Padico} et \emph{Adage} de l'\emph{INRIA.}\newline
  \begin{center}
   \begin{tabular}{|c|c|c|c|}
    \hline
    & ex32-12.gin & testTree100.gin & testTree1000.gin\\
    \hline
    Temps d'execution(en $\mu$s) & 0 & 1 & 2\\
    \hline
    Taille de la couverture & 0 & 1 & 2\\
    \hline
    \caption{Tableau comparatif des exécutions des plsuieurs exemples.\label{tableau}}
   \end{tabular}
  \end{center}
  