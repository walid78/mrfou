 \section{Bilan}
 Nous avons remarqué l'intérêt que peuvent avoir les algorithmes
 d'approximation lorsque le temps d'exécution est important. En effet,
 nous avons observé que l'algorithme du cours, même moins précis en
 général, est beaucoup plus rapide et avec la même certitude de
 $2$-approximation que l'algorithme du projet.\\

 Il est donc évident que sur des problèmes de très grandes tailles,
 lorsqu'il est impossible, au niveau temps, de calculer une solution
 exacte, on peut se tourner vers les algorithmes d'approximation qui
 sont d'une bien meilleure efficacité et qui donnent des solutions
 proches d'une solution optimale.