 \section{Organisation du projet}
 
  \subsection{Partie 1}
  Le but de cette partie est de trouver un algorithme qui calcule la
  couverture par sommets minimale d'un arbre en temps linéaire.

  Pour cela, nous allons montrer une propriété des couvertures par
  sommets sur les arbres, puis utiliser cette propriété pour proposer un
  algorithme linéaire de résolution du problème de minimisation d'une
  couverture par sommets sur un arbre.
  
  \subsection{Partie 2}
  Dans cette partie, on veut comparer deux algorithmes d'approximation
  pour la couverture par sommets :
  \begin{enumerate}
   \item L'algorithme du projet qui calcule un arbre correspondant à une
	 recherche en profondeur puis qui choisit comme couverture
	 l'ensemble des sommets qui ne sont pas des feuilles.
   \item L'algorithme du cours qui utilise le couplage maximal.
  \end{enumerate}

  Nous allons justifier que l'algorithme du projet est bien un
  algorithme $2$-approché du problème couverture par sommets et nous
  allons comparer les temps d'exécution et taille de la solution des
  deux algorithmes.

  \subsection{Organisation du rapport}
  Nous allons donc vous présenter dans ce rapport les étapes de la
  première partie du sujet, puis de la deuxième pour continuer par
  parler rapidement de l'implémentation du projet et enfin effectuer un
  bilan rapide.