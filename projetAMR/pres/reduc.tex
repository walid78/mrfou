%% reduc.tex %%
   \subsection{Instance positive}
   \begin{frame}
    \frametitle{4-Col}

    \begin{block}{\only<1,2>{Données}\only<3->{Solution}}
     \begin{center}
      \only<1,2>{
      \begin{tikz_mrfou}
       %% Nodes %%
       \node[gnode] (0) {0};
       \node[gnode, below left of=0] (1) {1};
       \node[gnode, below right of=1] (2) {2};
       \node[gnode, below right of=0] (3) {3};

       %% Edges %%
       \path[-] 
       (0)
       edge node {} (1)
       edge node {} (2)
       edge node {} (3)

       (1) 
       edge node {} (2)
       edge node {} (3)

       (2)
       edge node {} (3)

       ;
       
      \end{tikz_mrfou}
      }\only<3->{
      \begin{tikz_mrfou}
       %% Nodes %%
       \node[bluenode] (0) {0};
       \node[rednode, below left of=0] (1) {1};
       \node[greennode, below right of=1] (2) {2};
       \node[blacknode, below right of=0] (3) {3};

       %% Edges %%
       \path[-] 
       (0)
       edge node {} (1)
       edge node {} (2)
       edge node {} (3)

       (1) 
       edge node {} (2)
       edge node {} (3)

       (2)
       edge node {} (3)

       ;
       
      \end{tikz_mrfou}
      }
     \end{center}
    \end{block}

    \uncover<2->{
    \begin{block}{Question\only<3->{/Réponse}}
     Le graphe est-il 4-coloriable ? \only<3->{\alert<3->{Oui !}}
    \end{block}
    }
   \end{frame}

  \subsection{Instance négative}
  

  \subsection{Bonus: les ``cas faciles''}