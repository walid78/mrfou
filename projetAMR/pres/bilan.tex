%% bilan.tex %%
\begin{frame}
 \frametitle{Bilan}

 \begin{block}{Améliorations}
  \begin{itemize}
   \item Partie \emph{Réduction} : 
	 \begin{itemize}
	  \item Utiliser des structures de graphes différentes selon les
		réductions.
	  \item Trouver plus de \emph{``cas faciles''}
	 \end{itemize}
   \item<2-> Partie \emph{Approximation} :
	 \begin{itemize}
	  \item Utiliser une structure de graphe plus adaptée à la
		suppression des arêtes pour le couplage maximum.
	  \item Améliorer le \emph{parser} qui est la cause d'exécutions
		lentes.
	 \end{itemize}
  \end{itemize}
 \end{block}

 \begin{block}{Conclusion}<3->
  \begin{itemize}
   \item Nous avons remarqué l'intêret d'utiliser des algorithmes
	 d'approximation lorsque le calcul d'une solution exacte est
	 trop coûteux
   \item Un des rares projets où la théorie est au service de la
	 pratique
  \end{itemize}
 \end{block}
\end{frame}