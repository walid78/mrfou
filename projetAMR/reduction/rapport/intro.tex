 \section{Introduction}

  \subsection{Projet}
  L'objectif du projet est de réaliser un programme résolvant des
  problèmes \emph{NP-complets} sur des graphes non orientés.

  Pour ce faire, étant donné que le programme
  \emph{''Minisat''}\footnote{\url{http://minisat.se/}} sait résoudre
  ``efficacement'' des instances de \emph{SAT}, nous utilisons les
  réductions des problèmes qui nous intéressent au problème \emph{SAT}.

  \subsection{Organisation du rapport}
  Étant donné que le rapport doit être court, nous n'aurons
  malheureusement pas la possibilité d'expliquer la réalisation de notre 
  projet dans les détails. Aussi, nous laisserons une partie d'annexes
  où se trouveront plusieurs exemples intéressants que nous n'avons pas
  mis dans le rapport.

  Nous allons donc commencer par expliquer nos choix de programmation,
  puis allons donner les propriétés des problèmes qui nous ont permis de
  les réduire à \emph{SAT}. Nous vous présenterons ensuite quelques
  tests et résultats de notre programme, pour finir par un bilan de
  notre projet.