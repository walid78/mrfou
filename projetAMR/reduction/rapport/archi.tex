\subsection{Dépendances et Relations}

Le programme est simple d'utilisation puisqu'il suffit de le lancer avec un fichier d'entrée de type graphe issu du programme \emph{gengraph} de \emph{C. Gavoile} ainsi qu'avec le numéro du problème et éventuellement des paramètres propre au problème. \\
Le programme se déroule de la manière suivante :\\
\begin{itemize}
\item Le fichier de graphe est lu à partir du fichier de graphe fournit.
\item Selon le problème sélectionné, on lance la réduction correspondante.
\item Une formulation en clause de type SAT est créé et envoyé au SATSolver \emph{Minisat}.
\item \emph{Minisat} crée une solution (si celà est possible).
\item La solution fournit est analysé afin de pouvoir être traité. De cette manière, nous connaissons la solution au problème voulu sur le graphe passé en parmètre.
\end{itemize}
En décrivant l'exécution du programme de cette façon, il est aisé de
voir apparaître les relations entre les modules de notre programme. La
fonction principale \emph{main} contenu dans Solve.cpp joues le rôle du
trie des arguments et construction du graphe. Ensuite, est appelé la
réduction voulu contenu dans les fichiers portant le nom d'un problème
(ou leur abréviation). Une fois la réduction effectuée par ce fichier,
celui-ci appelle un parser pour \emph{Minisat} qui bien sur gère le fait
d'écrire un fichier d'entrée à \emph{Minisat}, lance le SATSolver et
analyse sa sortie. La sortie analysé, une assignation des variables est
créée, celle-ci nous permettra de donner l'ensemble des arêtes ou sommet
nécessaire pour la résolution du problème donné sur le graphe donné.