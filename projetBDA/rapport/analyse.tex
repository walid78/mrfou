\section{Analyse de notre base de donn\'ees}

\begin{table}[h]
%\begin{center}
\begin{tabular}{|l|l|l|l|}
\hline
\multicolumn{4}{|c|}{Calcul de l'occupation m\'emoire / physique}\\
\hline
Table& Poids des champs &Nombre de lignes&Poids total \\
\hline
Destination&number + 2*varchar2(20) = 22+2*20=62 octets&50&3ko\\
\hline
Circuit&number+varchar2(20)=22+20 = 42octets&3*50= 150 &6.3ko\\
\hline
Assoc\_D\_C&5*number+varchar2(20)+varchar2(50)=180o&3*50&6.6ko\\
\hline
H\^otel&5*number+varchar2(20)+varchar2(50)=180o&10 par circuit:10*3*50&270 ko\\
\hline
Classe\_H\^otel&number+2*float=22+44=66o&5&330o\\
\hline
S\'ejour&2*number+varchar2(50)+float= 44+50+22=116o&2&232o\\ 
\hline
Etape&2*number+varchar2(50)=44+50=94o&5*3*50=750&70.5\\
\hline
R\'eservation&2*number+2*date=44+14=58o&400p*3*50= 60k&3.48Mo\\
\hline
Assoc\_P\_S\_C&2*number+float=44+22=66o&2*150=300&19.8 Ko\\
\hline
Vol&2*number+2*float=22*4=88o&100 vols * 50 dest = 5k&440ko\\
\hline
Client&3*number+float+varchar2(50)+10+30+3*20=238o&400*12*10=48k&11.424Mo\\
\hline
Facturation&8*number+3*date+10+3*50+9*20+11*float=779o&400*12*10*5 etapes=240k&186.960.000 o\\
\hline
\end{tabular}
\end{table}


D\'ecoupage des tablespaces :
\begin{itemize}
\item De T1 \`a T10: La table de facturation segment\'ee par ann\'ee.
\item T11 : La table de LOG qui enregistre tous les \'ev\`enements sur la base.
\item T0: Toutes les autres tables.\\
\end{itemize}

Pour la facturation, nous avons compt\'e une moyenne de 5 h\^otels par r\'eservation ainsi qu'une ligne de vol, une ligne de circuit, et une ligne pour le total des prestations. Nous avons donc 779 octets par ligne , et 240k*(5 h\^otels + 3) lignes = 1.495.680.000 octets.\\
T1 \`a T10 = 1.495 Go\\
Total Occupation Tablespace T0 : 15,4 Mo (Arrondit à 20 Mo par mesure de pr\'ecaution)\\
Total Occupation Tablespace T11 estim\'e \`a 20 Mo (Stockage des \textit{logs}).

Chaque tablespace de T1 \`a T10 occuperont environ 149,5Mo.(Nous arrondirons \`a 200 Mo par s\'ecurit\'e pour chacun de nos tablespaces).
Il sera toujours possible \'a l'avenir d'adapter la taille des tablespaces en fonction de leur encombrement.
Les tableaux ci-dessous nous donnent les d\'etails du calcul de l'espace m\'emoire n\'ecessaire.

\newpage



