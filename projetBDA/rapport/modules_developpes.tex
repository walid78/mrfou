\section{Modules d\'evelopp\'es}
Nous avions comme premi\`ere intention de mettre au point un \textit{package} qui fournirait \`a notre application tous les \'el\'ements n\'ecessaires \`a son d\'eveloppement comme : \\

\begin{itemize}
\item Des proc\'edures et fonctions pour ajouter des \'el\'ements dans la base et calculer des statistiques.
\item Des variables: pour conna\^itre l'encombrement de chaque h\^otel par exemple.
\item Des exceptions: Entr\'ee d'une fiche erron\'ee, manque d'\'el\'ements pour une facture, ... \\
\end{itemize}

Nous n'avons pas r\'eussi \`a cr\'eer un tel \textit{package} pour des raisons que nous n'avons pas eu le temps d'identifier, nous avons donc seulement cr\'ee des proc\'edures et fonctions permettant de simplifier le c\^ot\'e applicatif du projet.

\subsection{Proc\'edure principale : \textit{AjoutFacture}}

\textit{AjoutFacture(client, s\'ejour, circuit, dest, vol, nombre\_adulte, nombre\_enfant)}\\ \\
Cette proc\'edure a \'et\'e d\'evelopp\'ee afin de faciliter le traitement au niveau de notre application.
Elle prend en param\`etre les diff\'erents identifiants n\'ec\'essaire \`a la saisie de la facture, ainsi que le nombre de personnes qui vont participer au voyage.\\

S\'equence de fonctionnement :
\begin{enumerate}[1]
\item R\'ecup\'eration des \'el\'ements dans les tables \`a partir des identifiants.
\item Parcours de la table \textit{r\'eservation}. Pour chaque ligne, on saisie la facture de l'h\^otel correspondant.
\item Remplissage des lignes 'Circuit', 'Vol', 'Total'.
\item Suppression des anciens \'el\'ements de la table r\'eservation.

\end{enumerate}

\subsection{Les fonctions PHP}
Nous présentons dans ce module les fonctions n\'ec\'essaires \`a la connection et d\'econnection \`a la base de donn\'ee, au listage de table et \`a l'ajout ou la suppression d'\'el\'ements.

\begin{enumerate}
\item Connect\_db() : Connection \`a la base sur le port 1521 avec les \textit{logins} et mot de passe ad\'equat.
\item Disconnect\_db() : D\'econnection de la base.
\item List\_table(table\_name): Afficher une table \`a l'\'ecran.
\item Traitement\_supp : Fonction permettant d'effacer d'une table les \'el\'ements selectionn\'es \`a l' \'ecran.
\item Traitement\_ajout : M\^eme principe mais pour l'ajout d'\'el\'ement.
\end{enumerate}

\subsection{Les triggers}
Quatre \textit{triggers} ont \'et\'e mis en place afin d'enregistrer les diff\'erentes op\'erations effectu\'ees par les op\'erateurs sur les tables des clients, des h\^otels et des circuits.
On peut donc retrouver dans la table LOG l'historique des diff\'erentes op\'erations avec le \textit{login} de l'op\'erateur, l'action qu'il a effectu\'ee (insertion, suppression, modification) et consulter la table sur laquelle il a agit ainsi que la date et l'heure de l'op\'eration.

\subsection{Les s\'equences}
Diff\'erentes s\'equences ont \'et\'e mises en place pour effectuer une num\'erotation automatique des diff\'erents identifiants, de fa\ con totalement transparente pour l'utilisateur.

\subsection{Les scripts \textbf{SQL}}
Dans cette section, nous décrivons les diff\'erents scripts qui sont contenus dans l'archive jointe : 
\begin{itemize}
\item \textit{creation\_table.sql} : Destruction et cr\'eation des \textit{Tablespaces} ainsi que des tables de notre base. 
\item \textit{procedures.sql} : Proc\'dures de remplissage simplifi\'e des tables ainsi que la proc\'edure \textit{AjoutFacture} vu pr\'ec\'edemment.
\item \textit{remplissage2.sql} : Destruction et cr\'eation des s\'equences, effacement des donn\'ees pr\'esentes dans les tables et utilisation des proc\'edues vues pr\'ec\'edemment pour remplir les tables.
\item \textit{trigger.sql} : Destruction et cr\'eation des \textit{triggers}.
\end{itemize}

A noter qu'il faut lancer les scripts dans l'ordre o\`u ils ont \'et\'e pr\'esent\'es.
\newpage