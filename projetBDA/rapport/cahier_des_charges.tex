\documentclass[10pt]{article}

\usepackage[utf8]{inputenc}
\usepackage[francais]{babel}

\author{Bruno Bassac, Geoffrey Graveaud, Fabien Kuntz}
\title{\textbf{Projet Master2 GL} \\
\textbf{Base de données avancées} \\
Gestion d'une agence de voyage}
\date{\today}

\begin{document}
\maketitle
\tableofcontents
\section{La présentation du projet}
présentation/introduction
les objectifs recherchés
la cible visée : grand public, professionnels, clientèle française, étrangère... (si on veut faire ça bien)

\section{Eléments principaux}
les éléments qui doivent être mis en avant : information, produits, services...
\section{Circuits de vente (optionnel)}
les circuits de vente : renvoi sur des agences, vente en ligne...
\section{Les tables :partie stockages des données}
la logistique : emplacement des stocks, du serveur, répartition géographique des clients...
Description de nos tables.
\section{Le site}
les fonctions du site.
\section{Les applications nécessaires}
les applications nécessaires.
\section{Description tecnhique de notre BDD}
l'arborescence et le nom des principales rubriques.
\section{Les modalités de mise à jour}
les modalités de mise à jour : la réalisation d'un "back office", outil qui permet d'effectuer soi-même les mises à jour du site.
En effet si le site n'évolue pas, n'est pas à jour, le retour d'image pour l'entreprise ne sera pas bon.
\end{document}  