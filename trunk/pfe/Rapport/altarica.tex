 \section{Noeuds AltaRica}
 
  \subsection{LightSensor}

  \subsubsection{Le code Altarica}
  \verbatiminput{../src/altarica/alt/LightSensor.alt}


  Ce noeud correspond au capteur de lumière, et possede une variable d'état nommée value qui peux prendre trois valeurs différentes, correspondants au trois couleurs de case que nous allons utiliser: Noir, gris et blanc.
  
  \subsubsection{La vérification acheck}

  \subsection{UltraSonicSensor}
  \subsubsection{Le code Altarica}
  \verbatiminput{../src/altarica/alt/UltraSonicSensor.alt}
  


  De la même maniere que le noeud LightSensor, UltraSonicSensor comporte une variable "signal", avec cinq valeurs possibles, qui servira au robot maitre à savoir a quelle distance se trouve le robot esclave.

  \subsubsection{La vérification acheck}
  

  \subsection{Motor}
  \subsubsection{Le code Altarica}
  \verbatiminput{../src/altarica/alt/Motor.alt}



  Les moteurs sont modélisés de façon à avoir seulement trois vitesses possible: Avancer, stopper, reculer.

  \subsubsection{La vérification acheck}

  \subsection{Moving}
  \subsubsection{Le code Altarica}
  \verbatiminput{../src/altarica/alt/Moving.alt}

  Moving est la pour faire la synchronisation entre les deux moteurs munis de roues d'un robot. Il y a cinq ordres possible, avancer, s'arreter, reculer, à droite et à gauche.

  \subsubsection{La vérification acheck}

 
  \subsection{BTMaster}

  \subsubsection{Le code Altarica}
  \verbatiminput{../src/altarica/alt/BTMaster.alt}

