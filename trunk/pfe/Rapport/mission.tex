
 \section{Mission}

  \subsection{Environnement}
  L'environnement est constitu\'{e} d'un marquage au sol. Ce marquage
  correspond \`{a} un chemin d\'{e}limit\'{e} par des feuilles de couleurs
  diff\'{e}rentes de celle du sol. Un point d'arriv\'{e}e est pr\'{e}d\'{e}fini. On peut
  repr\'{e}senter l'environnement par une grille sur laquelle les
  d\'{e}placements possibles sont verticaux ou horizontaux. Les \'{e}l\'{e}ments de
  marquage sont donc de forme carr\'{e}e et on d\'{e}finit un chemin unique
  repr\'{e}sentant la combinaison de d\'{e}placements que les robots doivent
  \^{e}tre capables de r\'{e}aliser. Les couleurs des cases de la grille sont
  altern\'{e}es de mani\`{e}re \`{a} obtenir un damier. Ce d\'{e}tail permet au robot de
  diff\'{e}rencier une impasse d'un chemin possible lorsqu'il \'{e}value les
  possibilit\'{e}s autour de sa position courante.

  \subsection{Entit\'{e}s}

   \subsubsection{Robot Ma\^{i}tre}
   Le robot ma\^{i}tre est le robot qui poss\`{e}de les capteurs. C'est
   \'{e}galement lui qui d\'{e}cide des actions que les deux robots doivent
   effectuer. Il est capable de donner des ordres au robot esclave
   gr\^{a}ce \`{a} des composants Bluetooth. Deux capteurs sont mont\'{e}s sur
   ce robot : un capteur de lumi\`{e}re \`{a} l'avant, et un capteur
   ultrasonique \`{a} l'arri\`{e}re. Les d\'{e}placements possibles du robot sont
   les suivants : avancer jusqu'\`{a} la d\'{e}tection d'un changement
   d'environnement, tourner \`{a} gauche ou \`{a} droite sur place \`{a}
   90 deg. Le
   capteur de lumi\`{e}re permet de d\'{e}tecter les bordures du marquage. Le
   capteur ultrasonique detecte la pr\'{e}sence du robot esclave \`{a}
   l'arri\`{e}re. Il est particuli\`{e}rement utile apr\`{e}s un changement de
   direction pour arr\^{e}ter le robot esclave \`{a} hauteur du virage. 

   \subsubsection{Robot Esclave}
   Ce robot ne comporte aucun capteur, et est compl\`{e}tement d\'{e}pendant du
   robot ma\^{i}tre, \`{a} savoir qu'il n'effectue une action que lorsqu'il
   re\c coit un message par la communication Bluetooth. Les possibilit\'{e}s de
   d\'{e}placement de ce robot sont les m\^{e}mes.

   \subsubsection{Int\'{e}ractions}
   Lors du d\'{e}placement dans l'environnement et de la d\'{e}couverte
   d'obstacles, le robot ma\^{i}tre fait des choix de d\'{e}placement et
   communique au robot esclave des ordres lui permettant de passer
   \'{e}galement ces obstacles:
   \begin{itemize}
    \item Le robot ma\^{i}tre avance jusqu'\`{a} la d\'{e}tection d'une bordure
    \item Si une bordure est d\'{e}tect\'{e}e, les d\'{e}cisions suivantes sont
	  prises:
    \begin{itemize}
     \item Le robot tourne \`{a} gauche \`{a} 90 deg.
     \item Si une bordure est pr\'{e}sente, il sait que le chemin qu'il doit
	   emprunter est celui de droite.
    \end{itemize}
   \end{itemize}

  \subsection{Objectif}
  Les robots sont cens\'{e}s se d\'{e}placer \`{a} l'int\'{e}rieur du marquage et
  atteindre le point d'arriv\'{e}e. Le robot ma\^{i} doit explorer le terrain
  et permettre au robot esclave de le suivre. 
