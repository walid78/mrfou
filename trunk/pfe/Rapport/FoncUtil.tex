\section{R�duction API NXC}
Pour la r�alisation du  projet nous utiliserons 4 cat�gories de fonction : celles relatives aux moteurs, aux capteurs , � la connexion Bluetooth et �ventuellement d'autres fonctions permettant la coh�sion de l'ensemble.

\subsection{Les moteurs : }
\begin{verbatim}
	OnFwd("port","power"); //permet de mettre en marche "avant" un moteur du robot 
	OnRev("port","power"); //sens inverse
	RotateMotor("port","speed","degrees"); //
	RotateMotorEx("port","speed","degrees","turnpct","sync","stop"); 
\end{verbatim}

\subsection{Les capteurs : }

\subsubsection{Type et mode des capteurs :}
\begin{verbatim}
	SetSensorLight("port"); //Lie le capteur de lumi�re � un "port"
	SetSensorSound("port"); //Lie le capteur ultrason � un "port"
	SetSensorMode("port","const mode"); //Configure le mode du capteur associ� au "port"
	SetInput("port","const field","value"); //Configure le type du capteur associ� au "port"
	SetSensorLowspeed("port"); //Configure le capteur en mode I2C
	ClearSensor("port"); //Efface la valeur d'un capteur associ� � un "port"

\end{verbatim}

\subsubsection{Information des capteurs :}
\begin{verbatim}
	Sensor("n"); //retourne la valeur d'un capteur
	SensorUS("n"); //retourne la valeur du capteur ultrason

\end{verbatim}

\subsection{La connexion Bluetooth : }
\begin{verbatim}
	BluetoothStatues("NUMBER"); //V�rifie la connexion
	RemoteStartProgram("connection", "filename");
	RemoteStopProgram("connection");

\end{verbatim}

\subsection{Autres fonctions : }
\begin{verbatim}
	Wait("time"); //Temps donn� � l'ex�cution d'une action
	off("port"); //Stop le moteur associ� au "port"
	Acquire("mutex"); 
	Release("mutex"); 
	Precedes("task1","task2","taskn"); //Permet l'ex�cution de taches simultan�

\end{verbatim}

\subsection{Les variables : }
\begin{verbatim}
	OUT_A; //port A
	OUT_AC; //port AC
	IN_1; //Input 1
	IN_2; //Input 2

\end{verbatim}