\section{R\'{e}duction API NXC}
Pour la r\'{e}alisation du  projet nous utiliserons 4 cat\'{e}gories de fonction : celles relatives aux moteurs, aux capteurs , à la connexion Bluetooth et \'{e}ventuellement d'autres fonctions permettant la coh\'{e}sion de l'ensemble.

\subsection{Les moteurs }
\begin{verbatim}
	OnFwd("port","power"); //permet de mettre en marche "avant" un moteur du robot 
	OnRev("port","power"); //sens inverse
	RotateMotor("port","speed","degrees"); //
	RotateMotorEx("port","speed","degrees","turnpct","sync","stop"); 
\end{verbatim}

\subsection{Les capteurs }

\subsubsection{Type et mode des capteurs}
\begin{verbatim}
	SetSensorLight("port"); //Lie le capteur de lumière à un "port"
	SetSensorSound("port"); //Lie le capteur ultrason à un "port"
	SetSensorMode("port","const mode"); //Configure le mode du capteur associé au "port"
	SetInput("port","const field","value"); //Configure le type du capteur associé au "port"
	SetSensorLowspeed("port"); //Configure le capteur en mode I2C
	ClearSensor("port"); //Efface la valeur d'un capteur associé à un "port"

\end{verbatim}

\subsubsection{Information des capteurs}
\begin{verbatim}
	Sensor("n"); //retourne la valeur d'un capteur
	SensorUS("n"); //retourne la valeur du capteur ultrason

\end{verbatim}

\subsection{La connexion Bluetooth }
\begin{verbatim}
	BluetoothStatues("NUMBER"); //Vérifie la connexion
	RemoteStartProgram("connection", "filename");
	RemoteStopProgram("connection");

\end{verbatim}

\subsection{Autres fonctions }
\begin{verbatim}
	Wait("time"); //Temps donné à l'exécution d'une action
	off("port"); //Stop le moteur associé au "port"
	Acquire("mutex"); 
	Release("mutex"); 
	Precedes("task1","task2","taskn"); //Permet l'exécution de taches simultané

\end{verbatim}

\subsection{Les variables }
\begin{verbatim}
	OUT_A; //port A
	OUT_AC; //port AC
	IN_1; //Input 1
	IN_2; //Input 2

\end{verbatim}