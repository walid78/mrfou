%%%%% Remove ? to make this file compilable %%%%%
% TO COMPILE %

%%%%%%%%%%                             %%%%%%%%%%
%%%%%%%%%% AUTO INSERTION D'UNE ENTETE %%%%%%%%%%
%%%%%%%%%%                             %%%%%%%%%%
%%%%%%%%%%     POUR UN FICHIER TEX     %%%%%%%%%%
%%%%%%%%%%                             %%%%%%%%%%

%%%%%%          %%%%%%
%%%%%% PACKAGES %%%%%%
%%%%%%          %%%%%%

\documentclass[8pt]{beamer}
\usepackage[utf8]{inputenc}
\usepackage[english]{babel}
%\usepackage[babel=true,kerning=true]{microtype} % probleme avec : tikz
\usepackage{graphicx}
\usepackage{listings}
\usepackage{color} 
\usepackage{enumerate}
\usepackage{amsfonts}
\usepackage{amssymb}
\usepackage{amsmath}
\usepackage{wasysym}
\usepackage{moreverb}
%\usepackage{chemarrow}
% \usepackage{tikz}
% \usetikzlibrary{positioning}
% \usetikzlibrary{automata}
% \usetikzlibrary{trees}
% \usetikzlibrary{arrows}
\usepackage{beamerthemesplit}
\usetheme{JuanLesPins}


%%%%%%              %%%%%%
%%%%%% MISE EN PAGE %%%%%%
%%%%%%              %%%%%%


%%%%% Environnements %%%%%

% \newenvironment{tikz_mrfou}{
% \begin{tikzpicture}[->,>=stealth,shorten >=1pt,auto,node
%  distance=1.5cm,on grid, semithick, minimum size=5mm,bend
%  angle=10,font=\small, initial text=, label distance=-1mm]
% }{\end{tikzpicture}}

\definecolor{turq}{rgb}{0,0.60,0.60}
\newcommand{\janob}{Jan Obdr\v{z}\`alek}
\newcommand{\coul}[1]{\textcolor{red}{#1}}
\newcommand{\turq}[1]{\textcolor{turq}{#1}}
\newcommand{\magenta}[1]{\textcolor{magenta}{#1}}
\newcommand{\blue}[1]{\textcolor{blue}{#1}}
\newcommand{\Cn}[1]{\textcolor{black}{#1}}

% \tikzstyle{evennode}=[circle,draw=blue!50,fill=blue!20,thick, inner
% sep=0pt,minimum size=6mm]

% \tikzstyle{oddnode}=[rectangle,draw=black!50,fill=black!20,thick, inner
% sep=0pt,minimum size=5mm]


% \tikzstyle{pointnode}=[circle,draw=black,fill=black, thick, inner
% sep=0pt,minimum size=1mm] 

%%%%%%          %%%%%%
%%%%%% DOCUMENT %%%%%%
%%%%%%          %%%%%%


\begin{document}

%%%%%%              %%%%%%
%%%%%% ZONE DE CODE %%%%%%
%%%%%%              %%%%%%

%% Style bloc %%
\setbeamertemplate{blocks}[rounded][shadow=true]
\setbeamercovered{dynamic}

%% Page de garde %%
\author[J. Boyer, D. Cransac, J-S. Dubernet, F. Kuntz - Encadrant :
A. Griffault]{Jérémy Boyer, Dorian Cransac, Jean-Sébastien Dubernet,
Fabien Kuntz\\
~\\
\small Encadrant : Alain Griffault}
\institute{\footnotesize Module PFE\\~\\Master S\&T
Informatique\\~\\Université Bordeaux I\\} 
\title{Projet de fin d'études : \textit{``AltaRica et les robots''}} 
\date{\footnotesize 26 mars 2009}

\setcounter{page}{1}

%% Debut %%
\frame{\titlepage}
\frame{\tableofcontents}

 \section{Introduction}
 %==============================================================================
 \frame{\tableofcontents[current]}
  %==============================================================================
 \begin{frame}
  \frametitle{Le projet}
  
  \begin{block}{Principe}
   \begin{itemize}
    \item Deux robots : Maître et Esclave
	  \uncover<2->{
    \item Robot Maître : capteurs et moteurs}
	  \uncover<3->{
    \item Robot Esclave : seulement moteurs}
	  \uncover<4->{
    \item Maître communique les ordres à l'Esclave par Bluetooth}
	  \uncover<5->{
    \item Esclave : pas de décision}
   \end{itemize}
  \end{block}

  \uncover<6->{
  \begin{block}{But}
   \begin{itemize}
    \item Accomplir une mission non triviale mettant en jeu les deux robots
	  \uncover<7->{
    \item Vérifier cette mission en modélisant en Altarica}
    \uncover<8->{
    \item Etudier la possibilité d'une generation de code automatique}
   \end{itemize}
  \end{block}
  }

 \end{frame}


 \section{Mission}
 %==============================================================================
 \frame{\tableofcontents[current]}
  \section{R\'{e}sum\'{e}}

 intro...

 \section{Mission}

  \subsection{Environnement}
  L'environnement est constitu\'{e} d'un marquage au sol. Ce marquage
  correspond \`{a} un chemin d\'{e}limit\'{e} par des feuilles de couleurs
  diff\'{e}rentes de celle du sol. Un point d'arriv\'{e}e est pr\'{e}d\'{e}fini. On peut
  repr\'{e}senter l'environnement par une grille sur laquelle les
  d\'{e}placements possibles sont verticaux ou horizontaux. Les \'{e}l\'{e}ments de
  marquage sont donc de forme carr\'{e}e et on d\'{e}finit un chemin unique
  repr\'{e}sentant la combinaison de d\'{e}placements que les robots doivent
  \^{e}tre capables de r\'{e}aliser. Les couleurs des cases de la grille sont
  altern\'{e}es de mani\`{e}re \`{a} obtenir un damier. Ce d\'{e}tail permet au robot de
  diff\'{e}rencier une impasse d'un chemin possible lorsqu'il \'{e}value les
  possibilit\'{e}s autour de sa position courante.

  \subsection{Entit\'{e}s}

   \subsubsection{Robot Ma\^{i}tre}
   Le robot ma\^{i}tre est le robot qui poss\`{e}de les capteurs. C'est
   \'{e}galement lui qui d\'{e}cide des actions que les deux robots doivent
   effectuer. Il est capable de donner des ordres au robot esclave
   gr\^{a}ce \`{a} des composants Bluetooth. Deux capteurs sont mont\'{e}s sur
   ce robot : un capteur de lumi\`{e}re \`{a} l'avant, et un capteur
   ultrasonique \`{a} l'arri\`{e}re. Les d\'{e}placements possibles du robot sont
   les suivants : avancer jusqu'\`{a} la d\'{e}tection d'un changement
   d'environnement, tourner \`{a} gauche ou \`{a} droite sur place \`{a}
   90 deg. Le
   capteur de lumi\`{e}re permet de d\'{e}tecter les bordures du marquage. Le
   capteur ultrasonique detecte la pr\'{e}sence du robot esclave \`{a}
   l'arri\`{e}re. Il est particuli\`{e}rement utile apr\`{e}s un changement de
   direction pour arr\^{e}ter le robot esclave \`{a} hauteur du virage. 

   \subsubsection{Robot Esclave}
   Ce robot ne comporte aucun capteur, et est compl\`{e}tement d\'{e}pendant du
   robot ma\^{i}tre, \`{a} savoir qu'il n'effectue une action que lorsqu'il
   re\c coit un message par la communication Bluetooth. Les possibilit\'{e}s de
   d\'{e}placement de ce robot sont les m\^{e}mes.

   \subsubsection{Int\'{e}ractions}
   Lors du d\'{e}placement dans l'environnement et de la d\'{e}couverte
   d'obstacles, le robot ma\^{i}tre fait des choix de d\'{e}placement et
   communique au robot esclave des ordres lui permettant de passer
   \'{e}galement ces obstacles:
   \begin{itemize}
    \item Le robot ma\^{i}tre avance jusqu'\`{a} la d\'{e}tection d'une bordure
    \item Si une bordure est d\'{e}tect\'{e}e, les d\'{e}cisions suivantes sont
	  prises:
    \begin{itemize}
     \item Le robot tourne \`{a} gauche \`{a} 90 deg.
     \item Si une bordure est pr\'{e}sente, il sait que le chemin qu'il doit
	   emprunter est celui de droite.
    \end{itemize}
   \end{itemize}

  \subsection{Objectif}
  Les robots sont cens\'{e}s se d\'{e}placer \`{a} l'int\'{e}rieur du marquage et
  atteindre le point d'arriv\'{e}e. Le robot ma\^{i} doit explorer le terrain
  et permettre au robot esclave de le suivre. 


 \section{Modèle AltaRica}
 %==============================================================================
 \frame{\tableofcontents[current]}
  \section{Noeuds AltaRica}
 
  \subsection{LightSensor}
  \verbatiminput{../src/altarica/LightSensor.alt}

  \subsection{UltraSonicSensor}
  \verbatiminput{../src/altarica/UltraSonicSensor.alt}

  \subsection{Motor}
  \verbatiminput{../src/altarica/Motor.alt}

  \subsection{Moving}
  \verbatiminput{../src/altarica/Moving.alt}


  
 \section{Implémentation}
 %==============================================================================
 \frame{\tableofcontents[current]}
  \begin{frame}
  \frametitle{Algorithme}
  
  \begin{block}{Présentation}
   \begin{itemize}
    \item Une boucle While
	  \uncover<2->{
    \item Une lecture permanente des couleurs du damier}
	  \uncover<3->{
    \item la connexion bluetooth}
	  \uncover<4->{
    \item Une action par couleur}
   \end{itemize}
  \end{block}

  \uncover<5->{
  \begin{block}{Principe : }
   \begin{itemize}
    \item Ligne droite
	  \uncover<6->{
    \item Obstacle simple de face}
	  \uncover<7->{
    \item Obstacle non-trivial (face et gauche)}
   \end{itemize}
  \end{block}
  }

 \end{frame}

 \begin{frame}
  \frametitle{Difficultés}

  \begin{block}{}
   \begin{itemize}
    \item Taille des robots
	  \uncover<2->{
    \item Taille des cases du  damier}
	  \uncover<3->{
    \item La lumière}
	  \uncover<4->{
    \item Passage Blanc/Noir}
   \end{itemize}
  \end{block}


 \end{frame}

 \section{Conclusion}
 %==============================================================================
 \frame{\tableofcontents[current]}


%%%%%%               %%%%%%
%%%%%% /ZONE DE CODE %%%%%%
%%%%%%               %%%%%%

\end{document}



%%%%%%           %%%%%%
%%%%%% /DOCUMENT %%%%%%
%%%%%%           %%%%%%




%%%%% Texte italique %%%%%
%  \textit{} 

%%%%% Liste %%%%
%%% Itemize %%%
%  \begin{itemize}
%   \item item1\\
%   \item item2
%  \end{itemize}

%%% Enumerate %%%
%  \begin{enumerate}
%   \item item1
%   \item item2
%  \end{enumerate}

%%%%% Tabulations %%%%%
%   \begin{tabbing}
%    XX\=XX\=\kill
%    \>(OrdresEnonce.v, ligne 244)\\
%    \>\>test2
%   \end{tabbing}
 
%%%%% Note de pied de page %%%%%
%  \footnote{test}

%%%%% Référence %%%%%
%   \label{ref}
%% Plus loin :
%   \ref{ref}
%   \pageref{ref}

%%%%% Code %%%%%
% \begin{lstlisting}
%      List<Integer> lexBFS2 = new ArrayList<Integer>();
%      lexBFS2.add(3);
%      lexBFS2.add(2);
%      lexBFS2.add(4);
%      lexBFS2.add(1);
%      lexBFS2.add(0);    
%      assertNotNull(lexBFS2);	
%      assertEquals(lexBFS2,Graphs.lexBFS(nogYComp));
%   \end{lstlisting}

%%%%% Inclure une figure %%%%%
%  \begin{figure}[!ht]
%   \begin{center}
%	\includegraphics[width=7cm]{figs/cours1/fig2.eps}
%	\caption{\emph{MT2} : Calcul non déterministe}
%   \end{center}
%  \end{figure}

%%%%% Dessiner une figure %%%%% 
%   \begin{center}
%    \begin{tikz_mrfou}

%     %% Nodes %%
%     \node[state, initial] (1) {$1$};
%     \node[state, right of=1] (2) {$2$};
%     \node[state, right of=2] (3) {$3$};

%     %% Edges %%
%     \path[->] 
%     (1) 
%     edge [loop above] node {b} ()
%     edge node {a} (2)

%     (2)
%     edge [loop above] node {b} ()
%     edge [bend left] node {a} (3)

%     (3)
%     edge [bend left] node {b} (2);

%    \end{tikz_mrfou}
%   \end{center}
