\section{Modules d\'evelopp\'es}
Nous avions comme premi\`ere intention de mettre au point un package qui fournirait \`a notre application tous les \'el\'ement n\'ecessaire \`a son developpement comme :

\begin{itemize}
\item Des proc\'edures et fonctions: pour ajouter des \'el\'ements dans la base,calculer des statistiques
\item Des variables: connaitre l'emcombrement de chaque hotel par exemple
\item Des exceptions: Quand une fiche erron\'ee veut \^etre entr\'ee, ou qu'il manque des \'el\'ements pour une facture ...
\end{itemize}

Nous n'avons pas r\'eussit \`a cr\'eer un tel package pour des raisons que nous n'avons pas eut le temps d'identifer, nous avons donc seulement cr\'e\'e des proc\'edures et fonctions permetant de simplifier le cot\'e applicatif du projet.

\subsection{AjoutFacture}

AjoutFacture(client,sejour,circuit,dest,vol,nombre\_adulte,nombre\_enfant)\\
Cette proc\'edure a \'et\'e d\'evelopp\'ee afin de faciliter le traitement au niveau de notre application.
Elle prends en parametre les diff\'erents identifiants nec\'essaire \`a la saisie de la facture, ainsi que le nombre de personnes qui vont participer au voyage.\\

S\'equence de fonctionnement :
\begin{enumerate}[1]
\item R\'ecup\'eration des \'el\'ements dans les tables \'a partir des identifiants
\item Parcours de la table r\'eservation.Pour chaque ligne, on saisie la facture correspondante
\item Remplissage des lignes 'Circuit','Vol','Total'
\item Suppression des \'el\'ements anciens de la table r\'eservation

\end{enumerate}

\subsection{Functions.php}
Nous trouvons dans ce module les fonctions n\'ecessaires \`a la connection et d\'econnection \`a la base de donn\'ee,de listage de table, d'ajout et de suppression d'\'el\'ements.

\begin{enumerate}
\item Connect\_db() : Connection \`a la base sur le port 1521.
\item Disconnect\_db() : D\'econnection de la base.
\item List\_table(table\_name): Afficher une table \`a l' \'ecran.
\item Traitement\_supp : Fonction permettant d'effacer d'une table les \'el\'ements selectionn\'es a l' \'ecran
\item Traitement\_ajout : M\^eme principe mais pour l'ajour d'\'el\'ement
\end{enumerate}

\subsection{Les triggers}
Quatre triggers ont \'et\'e mis en place afin d'enregistrer les differentes op\'erations effectu\'ees par les op\'erateurs sur les tables de clients, d'hotel, de circuits.
On peut donc retrouver dans la table LOG l'historiques des diff\'erentes op\'erations avec le login de l'op\'erateur, l'action qu'il a \'effectu\'e (insertion, suppression, modification), la table sur laquelle il a agit ainsi que la date et l'heure.

\subsection{Les s\'equences}
Diff\'erentes s\'equences ont \'et\'ees mises en place pour effectuer une num\'erotation automatique des diff\'erents identifiants,de façon totallement transparente pour l'utilisateur.



\normalsize



\newpage