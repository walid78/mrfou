\documentclass[10pt]{article}

\usepackage[utf8]{inputenc}
\usepackage[francais]{babel}

\author{Bruno Bassac, Geoffrey Graveaud, Fabien Kuntz}
\title{\textbf{Projet Master2 GL} \\
\textbf{Base de donn�es avanc�es} \\
Gestion d'une agence de voyage}
\date{\today}

\begin{document}
\maketitle
\tableofcontents
\section{La pr�sentation du projet}
pr�sentation/introduction
les objectifs recherch�s
la cible vis�e : grand public, professionnels, client�le fran�aise, �trang�re... (si on veut faire �a bien)

\section{El�ments principaux}
les �l�ments qui doivent �tre mis en avant : information, produits, services...
\section{Circuits de vente (optionnel)}
les circuits de vente : renvoi sur des agences, vente en ligne...
\section{Les tables :partie stockages des donn�es}
la logistique : emplacement des stocks, du serveur, r�partition g�ographique des clients...
Description de nos tables.
\section{Le site}
les fonctions du site.
\section{Les applications n�cessaires}
les applications n�cessaires.
\section{Description technique de notre BDD}
l'arborescence et le nom des principales rubriques.
\section{Les modalit�s de mise à jour}
les modalit�s de mise � jour : la r�alisation d'un "back office", outil qui permet d'effectuer soi-m�me les mises � jour du site.
En effet si le site n'�volue pas, n'est pas � jour, le retour d'image pour l'entreprise ne sera pas bon.
\end{document}  