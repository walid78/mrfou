\section{Les tables de la base de donn\'ees}
%Choix et justification des différents modules développés

\subsection{Pr\'esentation des tables}
Voici la pr\'esentation des diff\'erentes tables que nous avons \'etablie. Elles sont pr\'esent\'ees sous la forme \textit{Nom\_de\_la\_table(\'el\'ement1 type, \'el\'ement2 type, $\dots$}. Peu de description est fournie car, pour la plupart des tables, leur nom est vocateur. A noter que les tables ne sont pas du tout d\'efinitive et que ce qui est \'ecrit en \textit{italique} est optionnelle.\bigskip

\begin{itemize}
\item Destination(ID\_Dest integer,Nom\_Destination varchar2(20),Pays varchar2(20));\\
\item Hotel(ID\_Hotel integer, ID\_Etape integer,Nom\_Hotel varchar2(20),Addresse varchar2(20), ID\_Classe integer, capac\_S integer, capac\_D integer);\\
\item Classe\_Hotel(ID\_Classe integer, Prix\_S float, Prix\_D float);\\
\item Circuit(ID\_Circuit integer,Nom\_Circuit varchar2(20));\\
\item Assoc\_Dest\_Circuit(ID\_Dest integer, ID\_Circuit integer);\\
\item Etape(ID\_Etape integer,ID\_Circuit integer,Descriptif varchar2(50));\\
\item Sejour(ID\_Sejour integer, Duree integer, Description varchar2(50), Coeff float);\\
\item Assoc\_Prix\_Sejour\_Circuit(ID\_Circuit integer, ID\_Sejour integer, Prix float);\\
\item Vol(ID\_Vol integer, ID\_Dest integer, Prix\_Enfant float, Prix\_Adulte float);\\
\item Reservation(ID\_Client integer,ID\_Hotel integer,Date\_Reservation date);\\ 
\item Client(ID\_Client integer,Addresse varchar2(50),Tel varchar2(10), Nom varchar2(20), Prenom varchar2(20), Age integer, Email varchar2(30), Classe\_sociale varchar2(20), ID\_Dest\_Preferee integer, Investissement\_Moyen float);\\

\item Facturation(ID\_Facture integer, Date\_Facture date, Adresse\_Client varchar2(50), Tel varchar2(10), Nom varchar2(20), Prenom varchar2(20), Nom\_Dest varchar2(20), Pays\_Dest varchar2(20), Nom\_Hotel varchar2(20), Adresse\_Hotel varchar2(50), Classe\_Hotel number, Prix\_S integer, Prix\_D integer, Nom\_circuit varchar2(20), Duree\_sejour integer, Prix\_Circuit float, Prix\_Vol\_Enfant float, prix\_Vol\_Adulte float,Nb\_Adulte number,Nb\_Enfant number, Description\_Sejour varchar2(50), Coeff\_Sejour float, Total\_Vol float, Total\_Hotel float,  Total\_Circuit float, Total\_Facture float, Age integer, Classe\_sociale varchar2(20), Dest\_Pref varchar2(20), Invest\_Moyen float);\\
\end{itemize}
\newpage

\subsection{Exemples de tables}

%%%%%%%%%%% Destination %%%%%%%%%%%

\begin{table}[h]
\begin{center}
\begin{tabular}{|l|c|c|}
\hline
\multicolumn{3}{|c|}{Destination}\\
\hline
ID\_dest& Nom & Pays \\
\hline
1 & Bordeaux& France\\
\hline
2 & Valence& France\\
\hline
3 & Valence& Espagne\\
\hline
4 & Barcelone& Espagne\\
\hline
5 & Paris& France\\
\hline
\end{tabular}
\end{center}
\caption{Table Destination}
\end{table}

%%%%%%%%%%% Table Hotel %%%%%%%%%%%

\begin{table}[h]
\begin{center}
\begin{tabular}{|l|c|c|c|c|c|c|}
\hline
\multicolumn{7}{|c|}{Hotel}\\
\hline
ID\_hotel& ID\_Circuit& Nom & Adresse &ID\_classe & Capac\_S & Capac\_D  \\
\hline
1 & 1& California&rue de LA& 4 & 20 & 10\\
\hline
2 & 4& L'h\^ote&Rue du serveur& 2 & 10 & 10\\
\hline
3 & 3& Herie&Rue de la blague& 5 & 40 & 35\\
\hline
\end{tabular}
\end{center}
\caption{Table Hotel}
\end{table}

%%%%%%%%%%% Table Classe Hotel %%%%%%%%%%%

\begin{table}[h]
\begin{center}
\begin{tabular}{|l|c|c|}
\hline
\multicolumn{3}{|c|}{Classe\_Hotel}\\
\hline
ID\_classe& Prix\_S & Prix\_D \\
\hline
1 & 10.00& 20.00\\
\hline
2 & 25.00& 45.00\\
\hline
3 & 30.00& 60.00\\
\hline
4 &  45.00& 80.00\\
\hline
5 & 60.00& 80.00\\
\hline
\end{tabular}
\end{center}
\caption{Table Classe\_Hotel}
\end{table}

%%%%%%%%%%% Table Circuit %%%%%%%%%%%

\begin{table}[h]
\begin{center}
\begin{tabular}{|l|c|}
\hline
\multicolumn{2}{|c|}{Circuit}\\
\hline
ID\_circuit& nom \\
\hline
1 & Mont\_Fuji\\
\hline
2 & Catalogne\\
\hline
3 & Monaco\\
\hline
4 &  Hockenheim\\
\hline
\end{tabular}
\end{center}
\caption{Table Circuit}
\end{table}
\newpage

%%%%%%%%%%% Table Assoc_Dest_Circuit %%%%%%%%%%%

\begin{table}[h]
\begin{center}
\begin{tabular}{|l|c|}
\hline
\multicolumn{2}{|c|}{Assoc\_Dest\_Circuit}\\
\hline
ID\_Dest& ID\_Circuit \\
\hline
1 & 4\\
\hline
2 & 3\\
\hline
2 & 1\\
\hline
\end{tabular}
\end{center}
\caption{Table Assoc\_Dest\_Circuit}
\end{table}

%%%%%%%%%%% Table Sejour %%%%%%%%%%%

\begin{table}[h]
\begin{center}
\begin{tabular}{|l|c|c|c|}
\hline
\multicolumn{4}{|c|}{Sejour}\\
\hline
ID\_Sejour& Duree & Description& Coeff\\
\hline
1 & 10& Courte duree & 1.5\\
\hline
2 & 21& Longue duree &1.0\\
\hline
\end{tabular}
\end{center}
\caption{Table S\'ejour}
\end{table}

%%%%%%%%%%% Table Assoc_Prix_Sejour_Circuit %%%%%%%%%%%

\begin{table}[h]
\begin{center}
\begin{tabular}{|l|c|c|}
\hline
\multicolumn{3}{|c|}{Assoc\_Prix\_Sejour\_Circuit}\\
\hline
ID\_Circuit& ID\_Sejour & Prix\\
\hline
1 & 2& 100.00\\
\hline
2 & 2& 115.00\\
\hline
2 & 4& 50.00\\
\hline
4 & 3& 65.00\\
\hline
\end{tabular}
\end{center}
\caption{Table Assoc\_Prix\_S\'ejour\_Circuit}
\end{table}
\newpage
%%%%%%%%%%% Table Vol %%%%%%%%%%%

\begin{table}[h]
\begin{center}
\begin{tabular}{|l|c|c|c|}
\hline
\multicolumn{4}{|c|}{Vol}\\
\hline
ID\_Vol& ID\_dest & Prix\_Enfant& Prix\_Adulte\\
\hline
1 & 1& 10.00 & 20.00\\
\hline
2 & 1& 15.00& 40.00\\
\hline
3 & 2& 9.00& 21.00\\
\hline
4 & 4& 12.00& 24.00\\
\hline
\end{tabular}
\end{center}
\caption{Table Vol}
\end{table}

%%%%%%%%%%% Table Client %%%%%%%%%%%

\begin{table}[h]
ID est l'identifiant client,
C\_S est classe sociale,
ID\_DP est la destination pr\'ef\'er\'ee du client,
\textbf{Le nom de la destination pr\'ef\'er\'ee du client a \'et\'e enlev\'ee.}
I\_Moyen est l'investissement moyen du client,
\bigskip

\begin{tabular}{|l|c|c|c|c|c|c|c|c|c|}
\hline
\multicolumn{10}{|c|}{Client}\\
\hline
ID& Addr& Tel & Nom & Prenom & Age & Email&C\_S & ID\_DP &I\_Moyen\\
\hline
1 &Rue de la plage&0746573894&Dupont&Raoul&67 &cal@hotmail.fr&Retrait\'e&2&1859.87\\
\hline
2 &Rue du geek&0556654558&Dupont&Paul&21 &paul@zik.net&Etudiant&1&150.78\\
\hline
3 &Rue de la soif&0557348875&Durand&Patrick&30 &dudul@bad.com&Ing\'enieur&2&179.78\\
\hline
\end{tabular}
\caption{Table Client}
\end{table}

%%%%%%%%%%%Table Reservation%%%%%%%%%%%%%%%

\begin{table}[h]
La table \textbf{Reservation} est une table temporaire qui nous permet de pouvoir s\'electionner plusieurs h\^otels lors du choix d'un circuit pour un h\^otel par un client. Une fois la commande saisie et la facture entre dans la base, les lignes correspondant au client sont supprim\'ees.
\begin{center}
\begin{tabular}{|l|c|c|}
\hline
\multicolumn{3}{|c|}{Reservation}\\
\hline
ID\_Client& ID\_Hotel&Date\_Res\\
\hline
1 & 4&3/3/2008\\
\hline
1 & 3&9/8/2008\\
\hline
1 & 5&7/2/2008\\
\hline
2 & 1&5/3/2008\\
\hline
2&4&2/7/2008\\
\hline
2 & 2&1/5/2008\\
\hline
\end{tabular}
\end{center}
\caption{Table R\'eservation}
\end{table}

\begin{table}[h]
 La table \textbf{LOG} est une fonctionnalit\'e que nous avons int\'egr\'e dans notre application. Elle a pour but d'enregistrer les actions des op\'erateurs sur la base de donn\'ee (afin d'effectuer des statistiques bien sur ...)\\
Le type timestamp repr\'esentant \'a la fois la date et l'heure, ce champs nous servira de cl\'e primaire.
\begin{center}
\begin{tabular}{|l|c|c|c|}
\hline
\multicolumn{4}{|c|}{LOG}\\
\hline
madate&utilisateur&action&cible\\
\hline
1 &Lagarde C. 1&UPDATE&CLients\\
\hline
2 &Petit Nicolas&DELETE&R\'eservation\\
\hline
3 &Laporte Bernard&INSERT&Vol\\
\hline
\end{tabular}
\end{center}
\caption{Exemple de table LOG}
\end{table}
\newpage

