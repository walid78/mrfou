\documentclass[10pt]{article}

\usepackage[utf8]{inputenc}
\usepackage[francais]{babel}

\author{Bruno Bassac, Geoffrey Graveaud, Fabien Kuntz}
\title{\textbf{Projet Master2 GL} \\
\textbf{Base de données avancées} \\
Les tables}
\date{\today}

\begin{document}
\maketitle

\section{Présentation des tables}
Voici la présentation des différentes tables que nous avons établie. Elles sont présentées sous la forme \textit{Nom\_de\_la\_table(élément1 type, élément2 type, $\dots$}. Peu de description est fournie car, pour la plupart des tables, leur nom est évocateur. A noter que les tables ne sont pas du tout définitive et que ce qui est écrit en \textit{italique} est optionnelle.\bigskip

\begin{itemize}
<<<<<<< .mine
\item Destination(ID\_Dest integer,Nom varchar2(20),Pays varchar2(20));\\
\item Hotel(ID\_Hotel integer, ID\_Dest integer,Nom varchar2(20),Addresse varchar2(50), ID\_Classe integer, capac\_S integer, capac\_D integer);\\
\item Classe\_Hotel(ID\_Classe integer, Prix\_S integer, Prix\_D integer);\\
=======
\item Destination(ID\_Dest integer,Nom_Hotel varchar2(20),Pays varchar2(20));\\
\item Hotel(ID\_Hotel integer, ID\_Dest integer,Nom_Hotel varchar2(20),Addresse varchar2(20), ID\_Classe integer, capac\_S integer, capac\_D integer);\\
\item Classe\_Hotel(ID\_Classe integer, Prix\_S integer, Prix\_D integer);\\
>>>>>>> .r174
\item Circuit(ID\_Circuit integer,Nom\_Circuit varchar2(20));\\
\item Assoc\_Hotel\_Circuit(ID\_Hotel integer, ID\_Circuit integer);\\
\item Assoc\_Dest\_Circuit(ID\_Dest integer, ID\_Circuit integer);\\
\item Sejour(ID\_Sejour integer, Duree integer, Description varchar2(50), Coeff float);\\
\item Assoc\_Prix\_Sejour\_Circuit(ID\_Circuit integer, ID\_Sejour integer, Prix float);\\
\item Vol(ID\_Vol integer, ID\_Dest integer, Prix\_Enfant float, Prix\_Adulte float);\\
<<<<<<< .mine
\item Client(ID\_Client integer,Addresse varchar2(50),Tel varchar2(10), Nom varchar2(20), Prenom varchar2(20), Age integer, Email varchar2(30), Classe\_sociale varchar2(20), ID\_Dest\_Preferee integer, Investissement\_Moyen float);\\

\item Facturation(ID\_Facture integer, Date\_Facture date, Addresse\_Client varchar2(50), Tel varchar2(10), Nom varchar2(20), Prenom varchar2(20), Nom\_Dest varchar2(20), Pays\_Dest varchar2(20), Nom\_Hotel varchar2(20), Address\_Hotel varchar2(50), Classe\_Hotel varchar2(20), Prix\_S integer, Prix\_D integer, Nom\_circuit varchar2(20), Duree\_sejour integer, Prix\_Circuit float, Prix\_Vol\_Enfant float, prix\_Vol\_Adulte float, Description\_Sejour varchar2(50), Coeff\_Sejour float, Total\_Vol float, Total\_Hotel float,  Total\_Circuit float, Total\_Facture float, Age integer, Classe\_sociale varchar2(20), Dest\_Pref varchar2(20), Invest\_Moyen float);\\
=======
\item Client(ID\_Client integer,Addresse varchar2(50),Tel varchar2(10), Nom varchar2(20), Prenom varchar2(20),Email varchar2(30), Age integer, Classe\_sociale varchar2(20), ID\_Dest\_Preferee integer, Investissement\_Moyen float);\\

\item Facturation(ID\_Facture integer, Date\_Facture date, Addresse\_Client varchar2(50), Tel varchar2(10), Nom varchar2(20), Prenom varchar2(20), Nom\_Dest varchar2(20), Pays\_Dest varchar2(20), Nom\_Hotel varchar2(20), Address\_Hotel varchar2(50), Classe\_Hotel integer, Prix\_S integer, Prix\_D integer, Nom\_circuit varchar2(20), Duree\_sejour integer, Prix\_Circuit float, Prix\_Vol\_Enfant float, prix\_Vol\_Adulte float, Texte\_Description\_Sejour varchar2(50), Coeff\_Sejour float, Total\_Vol float, Total\_Hotel float,  Total\_Circuit float, Total\_Facture float, \textit{Age integer}, \textit{Classe\_sociale varchar2(20)}, \textit{Dest\_Pref varchar2(20)}, \textit{Invest\_Moyen float} );\\
>>>>>>> .r174
\end{itemize}

\section{Exemples de tables}

%%%%%%%%%%% Destination %%%%%%%%%%%

\begin{table}[h]
\begin{center}
\begin{tabular}{|l|c|c|}
\hline
\multicolumn{3}{|c|}{Destination}\\
\hline
ID\_dest& Nom & Pays \\
\hline
1 & Bordeaux& France\\
\hline
2 & Valence& France\\
\hline
3 & Valence& Espagne\\
\hline
4 & Barcelone& Espagne\\
\hline
5 & Paris& France\\
\hline
\end{tabular}
\end{center}
\caption{Table Destination}
\end{table}

%%%%%%%%%%% Table Hotel %%%%%%%%%%%

\begin{table}[h]
\begin{center}
\begin{tabular}{|l|c|c|c|c|c|c|}
\hline
\multicolumn{7}{|c|}{Hotel}\\
\hline
ID\_hotel& ID\_dest& Nom & Addresse &ID\_classe & Capac\_S & Capac\_D  \\
\hline
1 & 1& California&rue de LA& 4 & 20 & 10\\
\hline
2 & 4& L'hôte&Rue du serveur& 2 & 10 & 10\\
\hline
3 & 3& Herie&Rue de la blague& 5 & 40 & 35\\
\hline
\end{tabular}
\end{center}
\caption{Table Hotel}
\end{table}
\newpage

%%%%%%%%%%% Table Classe Hotel %%%%%%%%%%%

\begin{table}[h]
\begin{center}
\begin{tabular}{|l|c|c|}
\hline
\multicolumn{3}{|c|}{Classe\_Hotel}\\
\hline
ID\_classe& Prix\_S & Prix\_D \\
\hline
1 & 10.00& 20.00\\
\hline
2 & 25.00& 45.00\\
\hline
3 & 30.00& 60.00\\
\hline
4 &  45.00& 80.00\\
\hline
5 & 60.00& 80.00\\
\hline
\end{tabular}
\end{center}
\caption{Table Classe\_Hotel}
\end{table}

%%%%%%%%%%% Table Circuit %%%%%%%%%%%

\begin{table}[h]
\begin{center}
\begin{tabular}{|l|c|}
\hline
\multicolumn{2}{|c|}{Circuit}\\
\hline
ID\_circuit& nom \\
\hline
1 & Mont\_Fuji\\
\hline
2 & Catalogne\\
\hline
3 & Monaco\\
\hline
4 &  Hockenheim\\
\hline
\end{tabular}
\end{center}
\caption{Table Circuit}
\end{table}

%%%%%%%%%%% Table Assoc_Hotel_Circuit %%%%%%%%%%%

\begin{table}[h]
\begin{center}
\begin{tabular}{|l|c|}
\hline
\multicolumn{2}{|c|}{Assoc\_Hotel\_Circuit}\\
\hline
ID\_circuit& ID\_Hotel \\
\hline
1 & 2\\
\hline
1 & 2\\
\hline
3 & 3\\
\hline
4 & 1 \\
\hline
\end{tabular}
\end{center}
\caption{Table Assoc\_Hotel\_Circuit}
\end{table}
\newpage

%%%%%%%%%%% Table Assoc_Dest_Circuit %%%%%%%%%%%

\begin{table}[h]
\begin{center}
\begin{tabular}{|l|c|}
\hline
\multicolumn{2}{|c|}{Assoc\_Dest\_Circuit}\\
\hline
ID\_Dest& ID\_Circuit \\
\hline
1 & 4\\
\hline
2 & 3\\
\hline
2 & 1\\
\hline
\end{tabular}
\end{center}
\caption{Table Assoc\_Dest\_Circuit}
\end{table}

%%%%%%%%%%% Table Séjour %%%%%%%%%%%

\begin{table}[h]
\begin{center}
\begin{tabular}{|l|c|c|c|}
\hline
\multicolumn{4}{|c|}{Sejour}\\
\hline
ID\_Sejour& Duree & Description& Coeff\\
\hline
1 & 10& Cool & 1\\
\hline
2 & 10& Funny& 1\\
\hline
3 & 21& Chiant& 0.5\\
\hline
4 & 21& Fatiguant& 0.4\\
\hline
\end{tabular}
\end{center}
\caption{Table Séjour}
\end{table}

%%%%%%%%%%% Table Assoc_Prix_Séjour_Circuit %%%%%%%%%%%

\begin{table}[h]
\begin{center}
\begin{tabular}{|l|c|c|}
\hline
\multicolumn{3}{|c|}{Assoc\_Prix\_Sejour\_Circuit}\\
\hline
ID\_Circuit& ID\_Sejour & Prix\\
\hline
1 & 2& 100.00\\
\hline
2 & 2& 115.00\\
\hline
2 & 4& 50.00\\
\hline
4 & 3& 65.00\\
\hline
\end{tabular}
\end{center}
\caption{Table Assoc\_Prix\_Séjour\_Circuit}
\end{table}
\newpage
%%%%%%%%%%% Table Vol %%%%%%%%%%%

\begin{table}[h]
\begin{center}
\begin{tabular}{|l|c|c|c|}
\hline
\multicolumn{4}{|c|}{Vol}\\
\hline
ID\_Vol& ID\_dest & Prix\_Enfant& Prix\_Adulte\\
\hline
1 & 1& 10.00 & 20.00\\
\hline
2 & 1& 15.00& 40.00\\
\hline
3 & 2& 9.00& 21.00\\
\hline
4 & 4& 12.00& 24.00\\
\hline
\end{tabular}
\end{center}
\caption{Table Vol}
\end{table}

%%%%%%%%%%% Table Client %%%%%%%%%%%

\begin{table}[h]
ID\_C est l'identifiant client,
C\_S est classe sociale,
ID\_DP est la destination préférée du client,
\textbf{Le nom de la destination préférée du client a été enlevée.}
I\_Moyen est l'investissement moyen du client,
\bigskip

\begin{tabular}{|l|c|c|c|c|c|c|c|c|c|}
\hline
\multicolumn{10}{|c|}{Client}\\
\hline
ID\_C& Addr& Tel & Nom & Prenom & Age & Email&C\_S & ID\_DP &I\_Moyen\\
\hline
1 & & &&  &  & &&&\\
\hline
2 & & &&  &  & &&&\\
\hline
3 & & &&  &  & &&&\\
\hline
\end{tabular}
\caption{Table Client}
\end{table}

%%%%%%%%%%% Table Facturation %%%%%%%%%%%

\textbf{Pour la table facturation, je m'en sens pas le courage sur papier car elle est vraiment grande :) }

\section{Quelques idées optionnelles en plus}
Rajouter les \textit{Dates\_de\_Départ} et de \textit{Dates\_de\_Retour} ?? \\
Citer la \textit{ville de départ} ? \\
Mettre le \textit{budget\_du\_client} ? C'est optionnelle mais les agences de voyages le font. \\

<<<<<<< .mine
\textbf{Version revue le \today}
=======

>>>>>>> .r174
\end{document}