\section{Analyse de notre base de donn\'ees}
%Argumenter choix des tables(ordinaires,index, tables partitionnées,cluster,tables organisées en index et les param de stockage)
%Argumenter les tailles
%Justifier le développement des procédures stockées.

\begin{table}[h]
%\begin{center}
\begin{tabular}{|l|l|l|l|}
\hline
\multicolumn{4}{|c|}{Calcul de l'occupation m\'emoire / physique}\\
\hline
Table& Poids des champs &Nombre de lignes&Poids total \\
\hline
Destination&number + 2*varchar2(20) = 22+2*20=62 octets&50&3ko\\
\hline
Circuit&number+varchar2(20)=22+20 = 42octets&3*50= 150 &6.3ko\\
\hline
Assoc\_D\_C&5*number+varchar2(20)+varchar2(50)=180o&3*50&6.6ko\\
\hline
Hotel&5*number+varchar2(20)+varchar2(50)=180o&10 par circuit:10*3*50&270 ko\\
\hline
Classe\_Hotel&number+2*float=22+44=66o&5&330o\\
\hline
Sejour&2*number+varchar2(50)+float= 44+50+22=116o&2&232o\\ 
\hline
Etape&2*number+varchar2(50)=44+50=94o&5*3*50=750&70.5\\
\hline
Reservation&2*number+date=44+7=51o&400p*3*50= 60k&3.06Mo\\
\hline
Assoc\_P\_S\_C&2*number+float=44+22=66o&2*150=300&19.8 Ko\\
\hline
Vol&2*number+2*float=22*4=88o&100 vols * 50 dest = 5k&440ko\\
\hline
CLient&3*number+float+varchar2(50)+10+30+3*20=238o&400*12*10=48k&11.424Mo\\
\hline
Facturation&6*number+date+10+3*50+9*20+11*float=721o&400*12*10*5 etapes=240k&173.04 Mo\\
\hline
\end{tabular}
\end{table}


D\'ecoupage des tablespaces :
\begin{itemize}
\item De T1 \'a T10:La table de facturation segmente par ann\'ee
\item T11 :La table de LOG qui enregistre tous les \'ev\`enements sur la base
\item T0: Toutes les autres tables.
\end{itemize}

Pour la facturation, nous avons compt\'e une moyenne de 5 hotels par reservation ainsi qu'une ligne de vol, une ligne de circuit, et une ligne de total.Soit 721 octet par ligne , et 240k*(5 hotels + 3) = 1,384,320,000 octets.
Total Occupation Tablespace T0 : 15,4 Mo\\
T1 à T10 = 1.384 Go\\

Chaque Tablespace devrait donc faire environ 138Mo.(Nous arrondiront \`a 200 Mo par s\'ecurit\'e pour chacun de nos tablespaces).
Il sera toujours possible \'a l'avenir d'adapter la taille des tablespaces en fonction de leur encombrement.
Les tableaux ci-dessous nous donnent les d\'etails du calcul de l'espace m\'emoire n\'ecessaire.

\newpage



