%%%%% Remove ? to make this file compilable %%%%%
% TO COMPILE %

%%%%%%%%%%                             %%%%%%%%%%
%%%%%%%%%% AUTO INSERTION D'UNE ENTETE %%%%%%%%%%
%%%%%%%%%%                             %%%%%%%%%%
%%%%%%%%%%     POUR UN FICHIER TEX     %%%%%%%%%%
%%%%%%%%%%                             %%%%%%%%%%

%%%%%%          %%%%%%
%%%%%% PACKAGES %%%%%%
%%%%%%          %%%%%%

\documentclass[a4paper]{article}
\usepackage{fullpage}
\usepackage[utf8]{inputenc}
\usepackage[francais]{babel}
\usepackage{graphicx}
\usepackage{listings}
\usepackage{color} 
\usepackage{enumerate}
\usepackage{amsfonts}
\usepackage{amssymb}
\usepackage{amsmath}
\usepackage{wasysym}
\usepackage{chemarrow}
\usepackage[colorlinks=true, urlcolor=blue]{hyperref}
\usepackage[french,boxed,linesnumbered,ruled]{algorithm2e}
\usepackage{tikz}
\usetikzlibrary{positioning}
\usetikzlibrary{automata}

%%%%%%              %%%%%%
%%%%%% MISE EN PAGE %%%%%%
%%%%%%              %%%%%%


%%%%% Marges %%%%%
\addtolength{\textwidth}{-1cm}
\addtolength{\oddsidemargin}{-0.8cm}
\textheight=21cm % longueur utile de la page
\topmargin=0cm % marge haute
\headsep=40pt % séparation 20 points entre entête et texte
%\footskip=20pt % idem séparation pied de page

%%%%% Entetes et Enpieds de pages %%%%%
\usepackage{fancyhdr}
\pagestyle{fancy}
\lhead{\footnotesize{\textit{Bruno Bassac, Geoffrey Graveaud, Fabien
Kuntz}}}
\rhead{\footnotesize{\textit{Bases de données avancées - Michel
Prudence, Christian Rétoré}}}

%%%%% Numérotation sections %%%%
\setcounter{secnumdepth}{3}
\setcounter{tocdepth}{3}

%%%%% Types enumerate %%%%%
\let\enumeratesav\enumerate
\let\endenumeratesav\endenumerate

% \renewenvironment{enumerate}{
% \enumeratesav
%  \setlength{\itemsep}{0.5pt}
%  \setlength{\parskip}{0pt}
%  \setlength{\parsep}{0pt}}
% \endenumeratesav

\newenvironment{tikz_mrfou}{
\begin{tikzpicture}[->,>=stealth,shorten >=1pt,auto,node
 distance=2cm,on grid,semithick,inner sep=2pt,bend
 angle=30,font=\small, initial text=]
}{\end{tikzpicture}}

\renewcommand{\labelenumi}{\indent-}
\renewcommand{\labelenumii}{\indent$\bullet$}
\renewcommand{\labelenumiii}{\indent$\star$}
\renewcommand{\labelenumiv}{\indent\_}

%%%%% Titre %%%%%
\newlength{\larg}
\setlength{\larg}{14.5cm}

%%%Zone de Code%%%
\definecolor{gray}{gray}{0.86}  

\lstset{numbers=left, tabsize=2, frame=single, breaklines=true,
basicstyle=\ttfamily,numberstyle=\tiny\ttfamily, framexleftmargin=13mm,
backgroundcolor=\color{gray}, xleftmargin=14mm} 

\newsavebox{\fmbox}
\newenvironment{fmpage}[1]{\begin{lrbox}{\fmbox}\begin{minipage}{#1}}{\end{minipage}\end{lrbox}\fbox{\usebox{\fmbox}}}


%%%%%%          %%%%%%
%%%%%% DOCUMENT %%%%%%
%%%%%%          %%%%%%


\begin{document}


%%%%% Page de présentation %%%%%
\thispagestyle{empty}

\setlength{\unitlength}{1in}

%%%%% Image de fond %%%%%
%\begin{picture}(0,0)
% \put(0.2,-7){\includegraphics[width=\textwidth]{images/bx1_trans.eps}}
%\end{picture}

\begin{flushright}
 \noindent {\rule{\larg}{0.5mm}}
\end{flushright}
\vspace{7mm}
\begin{flushright}
 \Huge{\bf Bases de données avancées} \\
 \Huge{\bf Projet} \\
 ~\\
 \huge{Gestion d'une agence de voyage}\\
\end{flushright}
\vspace{7mm}
\begin{flushright}
 {\rule{\larg}{0.5mm}}
\end{flushright}
\vspace{2mm}
\begin{flushright}
 \large{\bf Professeurs : Michel Prudence, Christian Rétoré} \\
 ~\\
 \large{Master 2 Informatique}\\
 ~\\
 % \today
 \vspace{9cm}
 \large{Bruno Bassac, Geoffrey Graveaud, Fabien Kuntz}
{\rule{\larg}{0.5mm}}
\end{flushright}

\newpage

\addtolength{\oddsidemargin}{1cm}

%%%%% Table des matières %%%%%
\thispagestyle{empty}
\tableofcontents
\newpage

\setcounter{page}{1}


%%%%%%              %%%%%%
%%%%%% ZONE DE CODE %%%%%%
%%%%%%              %%%%%%

% \include{01-}

%%%%%%               %%%%%%
%%%%%% /ZONE DE CODE %%%%%%
%%%%%%               %%%%%%




\end{document}



%%%%%%           %%%%%%
%%%%%% /DOCUMENT %%%%%%
%%%%%%           %%%%%%




%%%%% Texte italique %%%%%
%  \textit{} 

%%%%% Liste %%%%
%%% Itemize %%%
%  \begin{itemize}
%   \item item1\\
%   \item item2
%  \end{itemize}

%%% Enumerate %%%
%  \begin{enumerate}
%   \item item1
%   \item item2
%  \end{enumerate}

%%%%% Tabulations %%%%%
%   \begin{tabbing}
%    XX\=XX\=\kill
%    \>(OrdresEnonce.v, ligne 244)\\
%    \>\>test2
%   \end{tabbing}
 
%%%%% Note de pied de page %%%%%
%  \footnote{test}

%%%%% Référence %%%%%
%   \label{ref}
%% Plus loin :
%   \ref{ref}
%   \pageref{ref}

%%%%% Code %%%%%
% \begin{lstlisting}
%      List<Integer> lexBFS2 = new ArrayList<Integer>();
%      lexBFS2.add(3);
%      lexBFS2.add(2);
%      lexBFS2.add(4);
%      lexBFS2.add(1);
%      lexBFS2.add(0);    
%      assertNotNull(lexBFS2);	
%      assertEquals(lexBFS2,Graphs.lexBFS(nogYComp));
%   \end{lstlisting}

%%%%% Inclure une figure %%%%%
%  \begin{figure}[!ht]
%   \begin{center}
%	\includegraphics[width=7cm]{figs/cours1/fig2.eps}
%	\caption{\emph{MT2} : Calcul non déterministe}
%   \end{center}
%  \end{figure}

%%%%% Dessiner une figure %%%%% 
% \begin{tikzpicture}[->,>=stealth,shorten >=1pt,auto,node
%  distance=2cm,on grid,semithick,inner sep=2pt,bend
%  angle=45,font=\small]
%  \node[initial,state] (1) {$q_1$}
%  edge [loop above] node {b} ();
%  \node[state] (2) [right=of 1] {$q_2$}
%  edge [<-]  node {a} (1)
%  edge [loop above] node {b} ();
%  \node[state,right=of 2] (3) {$q_3$}
%  edge [->, bend right] node {b} (2)
%  edge [<-, bend left] node {a} (2);
%  \node[state,below=of 2,accepting] (4) {$q_4$}
%  edge [<-] node {b} (2)
%  edge [loop right] node {b} ();
% \end{tikzpicture}
