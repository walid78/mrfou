 \section{Idées des réductions}
 Nous allons dans cette partie vous présenter très brièvement les idées
 de réductions que nous avons utilisées. Nous allons seulement citer les
 quelques propriétés, caractérisant nos problèmes, qui nous ont permis
 de générer les formules \emph{CNF}.
 
 Nous citerons aussi ce que nous avons appelé les ``cas
 faciles'', c'est à dire les cas où nous n'avons pas besoin de
 \emph{minisat} pour répondre à nos problèmes (problème de décision).

  \subsection{\emph{K-Col} $\leq_P$ \emph{SAT}}
  \underline{Propriétés caractérisant \emph{K-Col} :}
  \begin{enumerate}
   \item Chaque sommet est colorié
   \item Il n'y a qu'une seule couleur par sommet
   \item Deux sommets qui se suivent n'ont pas la même couleur
  \end{enumerate}

  \underline{Cas faciles :}
  \begin{enumerate}
   \item Le nombre de couleurs est négatif
   \item Le nombre de couleurs est supérieur au nombre de sommets
  \end{enumerate}

  \subsection{\emph{Circuit Hamiltonien} $\leq_P$ \emph{SAT}}
  \underline{Propriétés caractérisant \emph{Circuit Hamiltonien (HC)} :}
  \begin{enumerate}
   \item Chaque noeud est dans le circuit
   \item Un noeud a exactement une position dans le circuit sauf pour le
	 premier/dernier
   \item Deux sommets non voisins ne peuvent pas se suivre dans le
	 circuit
   \item Deux sommets n'ont pas la même position dans le circuit
   \item Le dernier sommet du circuit doit aussi être le premier
  \end{enumerate}

  \underline{Cas faciles :}
  \begin{enumerate}
   \item Le nombre d'arêtes du graphe est strictement inférieur au
	 nombre de sommets
   \item Le graphe est un graphe complet $\Rightarrow$ Toujours vrai
  \end{enumerate}

  \subsection{\emph{Clique} $\leq_P$ \emph{SAT}}
  \underline{Propriétés caractérisant \emph{Clique} :}
  \begin{enumerate}
   \item Chaque noeud est dans la clique
   \item Un noeud a exactement une position dans la clique
   \item Deux sommets non voisins ne peuvent pas être ensemble dans la
	 clique
   \item Deux sommets n'ont pas la même position dans la clique
  \end{enumerate}

  \underline{Cas faciles :}
  \begin{enumerate}
   \item \emph{Clique} de taille négative $\Rightarrow$ Erreur en entrée
   \item \emph{Clique} de taille 0 $\Rightarrow$ Toujours vrai
   \item \emph{Clique} de taille 1 $\Rightarrow$ Nécessite 1 seul sommet
   \item \emph{Clique} de taille 2 $\Rightarrow$ Nécessite 1 seule arête
   \item \emph{Clique} de taille strictement supérieure au nombre de
	 sommets $\Rightarrow$ Impossible
   \item A partir d'un certain nombre d'arêtes par rapport à un nombre
	 de sommets, il est obligatoire d'obtenir une clique d'une
	 certaine taille
   \item Il faut un nombre d'arêtes minimum pour former une clique d'une
	 certaine taille
  \end{enumerate}

  \subsection{\emph{Ensemble indépendant} $\leq_P$ \emph{SAT}}
  \underline{Propriétés caractérisant \emph{Ensemble Indépendant (IS)} :}
  \begin{enumerate}
   \item Inutiles car il y a une équivalence pour Ensemble indépendant
	 de taille $k$ sur $G = (V,E)$ et \emph{Clique} de taille $k$
	 sur $\overline{G}$
  \end{enumerate}

  \underline{Cas faciles :}
  \begin{enumerate}
   \item \emph{IS} de taille négative $\Rightarrow$ Erreur
   \item \emph{IS} de taille $0$ $\Rightarrow$ Toujours vrai
   \item \emph{IS} de taille $1$ $\Rightarrow$ Nécessite 1 seul sommet
   \item \emph{IS} de taille strictement supérieure au nombre de sommets
	 $\Rightarrow$ Impossible
   \item On a forcément un \emph{IS} d'une certaine taille si le nombre
	 d'arêtes est trop faible
   \item Pour que $k$ sommets puissent former un \emph{IS} dans un
	 graphe de taille $n$, il ne faut pas que le nombre d'arêtes
	 dépasse un certain nombre.
  \end{enumerate}

  \subsection{\emph{Couverture par sommets} $\leq_P$ \emph{SAT}}
  \underline{Propriétés caractérisant \emph{Couverture par sommets (VC)}
  :}
  \begin{enumerate}
   \item Inutiles car il y a une équivalence pour \emph{Couverture par
	 sommets} de taille $k$ sur $G = (V,E)$ et \emph{Clique} de
	 taille $|V|-k$ sur $\overline{G}$
  \end{enumerate}

  \underline{Cas faciles :}
  \begin{enumerate}
   \item \emph{VC} de taille négative $\Rightarrow$ Erreur
   \item \emph{VC} de taille $0$ $\Rightarrow$ Vrai s'il n'y a aucune
	 arête
   \item \emph{VC} de taille strictement supérieure au nombre de sommets
	 $\Rightarrow$ Impossible 
   \item Pour que k sommets puissent représenter une \emph{VC} dans un
	 graphe de taille $n$, il ne faut pas que le nombre d'arêtes
	 dépasse un certain nombre.
  \end{enumerate}

