 \section{Tests et résultats}
  Pour vérifier le bon fonctionnement de notre programme, nous avons
  effectué plusieurs tests.
  \subsection{Tests}
  Nous avons fait plusieurs tests sur différents types de fichiers
  d'entrés. Ces fichiers tests sont disponibles dans le répertoire
  \emph{jeux}. Nous avons testé notre progrmme sur des fichiers vides,
  et différents types de graphes. Des graphes à une seule arête, et des
  graphes générés par le programme \emph{gengraph}: des cliques à 100
  sommets, un chemin à 100 sommets, un graphe complet de 100 sommets, un
  arbre à 100 sommets ...
  \subsection{Résultats des tests}
  Ces tests nous ont permis de mettre en avant quelques erreurs de
  programmation que nous avons corrigés mais surtout de trouver des cas
  simples, dont nous avons parlé dans les parties précédentes, et de
  voir les limites des réductions et de \emph{minisat}.\newline
  \begin{itemize}
   \item \emph{Circuit Hamiltonien:} Si on lançait la recherche d'un
	 Circuit Hamiltonien sur un graphe complet, à cause du grand
	 nombre de clauses, le temps d'exécution de \emph{minisat}
	 explose alors qu'il n'y a rien de plus simple à trouver. Cas
	 que nous avons résolue mais si on enlève une arête à un graphe
	 complet le problème perciste.\\
   \item \emph{Couverture de sommet:} Si on lance ces deux
	 réductions sur un graphe de type chemin de taille 100 et que
	 l'on cherche une couverture de sommet de taille 50 ou
	 inférieur (qui n'est pas satisfaisable dans ce denier cas) le
	 temps d'exécution explose litéralement alors quen TD, nous
	 avons vu que pour ce genre de graphe il existe une ``formule''
	 et que pour un nombre de sommet paire $n$, la plus petite
	 couverture de sommet est de taille $n/2$ (voir Annexe 1).
  \end{itemize}
  Sur des graphes de taille correcte à l'echelle papier (i.e. 4 ou 5
  sommets), les resulats ont tous été concluant comme espéré (voir
  Annexe 2 et plus).