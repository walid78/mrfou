 \section{Bilan}
 
  \subsection{Difficultés rencontrées}
  Tout au long de notre projet, nous nous sommes heurtés à des
  difficultés.

  Nous avons eu quelques soucis de démarrage avec ce que l'on appelle
  communément la barrière du langage. En effet, n'ayant jamais créé un
  programme en \emph{C++}, les évidences, pour nous, n'en étaient
  pas. Les pointeurs que nous avions gentiment mis de côté dans notre
  esprit se sont rappelés à nous via des erreurs de
  compilation\footnote{Ce qui, vous en conviendrez, n'est pas la plus
  agréable façon de se retrouver.}. Cela nous a suivi tout au long de
  notre développement et continuera, à l'évidence, à nous suivre dans la
  suite du projet. 

  Au niveau des réductions, nous avons eu quelques problèmes, mais rien
  d'anormal. Nous avons de temps en temps oublié quelques propriétés
  mais nous nous en rendions assez rapidement compte en faisant tourner
  le programme sur des exemples.\\
  Néanmoins, \emph{Circuit Hamiltonien} nous a pris à revers dans la
  dernière ligne droite. En effet, nous n'avions pas considéré le cas du
  graphe formé de deux losanges rejoints par une arête (voir
  \ref{an_warningCH} page \pageref{an_warningCH} en annexe). Une refonte
  totale de notre réduction a dû être opérée car nous avions utilisé des
  variables sur les arêtes, ce qui n'éliminait pas le problème. C'est en
  nous inspirant de la réduction de \emph{Clique} à \emph{SAT} que nous
  avons refait celle pour \emph{Circuit Hamiltonien}.

  Nous avons eu aussi des problèmes au niveau de la concordance des
  différents emplois du temps qui fut pour nous
  \emph{NP-difficile}. Néanmoins ce problème n'était pas à l'insu de
  notre plein gré, car la composition du groupe ne nous était pas
  imposée.

  Dernier problème et non le moindre, la contrainte de temps et le
  chevauchement des projets. En effet, nous n'avons pas pu faire tout ce
  que nous aurions aimé réaliser. Nous allons donc vous donner quelques
  idées d'améliorations que nous avons eues mais que nous n'avons pas pu
  inclure dans notre programme.
  
  \subsection{Améliorations}
  Nous allons rapidement vous donner une courte liste des choses que
  nous aurions aimé mettre en place dans le programme mais n'ont pas pu
  l'être :
  \begin{enumerate}
   \item Vérifier les formats des fichiers en entrée\footnote{Nous avons
	 remarqué que si \emph{Solve.cpp} était un graphe, alors il
	 admettrait une \emph{Couverture par sommets} de taille $1$.}. 
   \item Trouver plus de ``cas faciles''.
   \item Donner le choix d'afficher ou pas l'exécution de
	 \emph{Minisat}. Il faudrait gérer l'option \emph{-v} qui, si
	 elle présente, permettrait d'afficher l'exécution de Minisat,
	 et sinon, écrirait cette exécution dans \emph{graph.min} (il
	 faut utiliser une redirection de la sortie d'erreur de
	 \emph{Minisat} ($2>$)).
   \item Gérer les envois de signaux. Typiquement, durant l'exécution de
	 \emph{Minisat}, si on fait un \emph{C-c}, le résultat affiché
	 est quelque peu effrayant.
   \item Utiliser des structures de graphe différentes selon les
	 réductions.
  \end{enumerate}
  
  \subsection{Conclusion}
  Tout d'abord, avant de conclure sur le travail effectué, nous tenons à
  exprimer le fait que nous sommes déçus de ne pas avoir pu expliquer
  plus en détail les raisonnements qui nous ont faits avancer dans le
  projet, notamment en ce qui concerne les réductions. Nous trouvons
  dommage que notre rapport doive se réduire à à peu près cinq
  pages\footnote{Rapport $\leq_P$ Cinq-pages}.

  Finalement, grâce au travail que nous avons effectué, nous avons
  compris l'utilité des réductions dans la pratique. Nous avons pu
  toucher du doigt la \emph{NP-complétude} et entrevoir les limites à la
  fois de nos réductions et du \emph{SAT-Solver Minisat}.