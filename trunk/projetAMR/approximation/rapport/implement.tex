 \section{Implémentation}

  \subsection{Réutilisation de code}
  Nous avons réutilisé la classe que nous avions implémentée dans le
  premier projet pour représenter les graphes ainsi que la partie
  \emph{''parser''} de la classe principale.

  Il nous restait donc à implémenter la classe d'arbre.

  \subsection{Implémentation de la classe d'arbre}
  Nous avons donc créé la classe \emph{Tree.cpp} qui dépend de
  \emph{Graph.cpp} où un arbre est créé à partir d'un parcours en
  profondeur sur le graphe passé en paramètre.

  \subsection{Algorithmes}
  L'algorithme de la partie $1$ a été implémenté dans \emph{Tree.cpp}
  dans la méthode \emph{coverTree}.

  L'algorithme du projet (partie $2$) a été implémenté dans
  \emph{Graph.cpp} dans la méthode \emph{coverProject}.

  L'algorithme du cours (partie $2$) a été implémenté dans
  \emph{Graph.cpp} dans la méthode \emph{coverCourses}.

  \subsection{Exécution du programme}
  Le programme est exécuté à partir de la commande \emph{./cover}, une
  fonction d'usage est appelée pour donner plus d'informations sur
  l'exécution.