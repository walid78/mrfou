  \subsection{Question 2}
  
   \subsubsection{Idée de l'algorithme}
   Afin de calculer une couverture de taille minimale sur un arbre $T$,
   nous utilisons un parcours en profondeur de l'arbre.\\
   En utilisant la Question 1 (\ref{part1q1} page \pageref{part1q1}),
   nous marquons les sommets qui sont parents d'une ou plusieurs
   feuilles car les feuilles sont de degré $1$. Ensuite, lors de la
   partie ``retour'' de la récursion, nous allons marqué les sommets
   qui sont pères de sommets non marqués.\\

   Pour qu'un sommet sache s'il doit être marqué, chacun de ses fils va
   lui dire \emph{''Marque toi !''}($true$) ou \emph{''Je suis déjà
   marqué donc tu n'as pas besoin de l'être vis-à-vis de
   moi.''}($false$).

   Marquer un sommet revient à mettre la case qui lui correspond dans
   le tableau de booléen $cover$ à $true$.

   \subsubsection{Algorithme}
   \begin{algorithm}[!ht]
     \SetLine
     $\mathbf{global~var}$ $cover$ : Boolean table;\\
     \Begin{
       \ForEach{$v \in V$}{
         $cover[v] := false$;
       }
       coverTree\_aux($root$);\\
       ~\\
       \Return{$cover$};
     }
     \caption{coverTree(): Boolean table}
   \end{algorithm}

   \begin{algorithm}[!ht]
     \SetLine
     $\mathbf{local~var}$ $markFather := true$ : Boolean;\\
     \Begin{
       \If{$v$ is a leaf}{
         \Return{$true$};
       }
       \ForEach{$(v,w) \in E$}{
         \If{coverTree\_aux($w$)}{
           $cover[node] := true$;\\
           $markFather := false$;
         }
       }
       ~\\
       \Return{$markFather$};
     }
     \caption{coverTree\_aux(Vertex $v$): Booléen}
   \end{algorithm}
   
   \subsubsection{Exemple}
   \tikzstyle{bluenode}=[circle,draw=blue!50,fill=blue!20,thick, inner
   sep=0pt,minimum size=6mm]

   \tikzstyle{rednode}=[circle,draw=red!50,fill=red!20,thick, inner
   sep=0pt,minimum size=6mm]
    
   \begin{center}
    \begin{tikz_mrfou}

     %% Nodes %%
     \node[bluenode] (0) {$0$};
     \node[bluenode, above right of=0] (1) {$1$};
     \node[bluenode, below right of=1] (2) {$2$};
     \node[bluenode, below right of=0] (3) {$3$};

     %% Edges %%
     \path[-]

     (0)
     edge node {} (1)
     edge node {} (2)
     edge node {} (3)
     
     (1) 
     edge node {} (2)

     (2)
     edge node {} (3)
     ;

    \end{tikz_mrfou}
   \end{center}

   À partir du graphe, on calcule un arbre correspondant par un
   parcours en profondeur à partir du sommet $0$ (à l'aide du
   constructeur de \emph{Tree.cpp}).

   \begin{center}
    \begin{tikz_mrfou}

     %% Nodes %%
     \node[bluenode] (0) {$0$};
     \node[bluenode, above right of=0] (1) {$1$};
     \node[bluenode, below right of=1] (2) {$2$};
     \node[bluenode, below right of=0] (3) {$3$};

     %% Edges %%
     \path[-]

     (0)
     edge node {} (1)
     
     (1) 
     edge node {} (2)

     (2)
     edge node {} (3)
     ;

    \end{tikz_mrfou}
   \end{center}
   On applique alors notre algorithme.

   \begin{center}
    \begin{tikz_mrfou}

     %% Nodes %%
     \node[bluenode] (0) {$0$};
     \node[bluenode, above right of=0] (1) {$1$};
     \node[bluenode, below right of=1] (2) {$2$};
     \node[bluenode, below right of=0] (3) {$3$};

     %% Edges %%
     \path[-]

     (0)
     edge [->, blue] node {} (1)
     
     (1) 
     edge node {} (2)

     (2)
     edge node {} (3)
     ;

    \end{tikz_mrfou}
    \begin{tikz_mrfou}

     %% Nodes %%
     \node[bluenode] (0) {$0$};
     \node[bluenode, above right of=0] (1) {$1$};
     \node[bluenode, below right of=1] (2) {$2$};
     \node[bluenode, below right of=0] (3) {$3$};

     %% Edges %%
     \path[-]

     (0)
     edge [->, blue] node {} (1)
     
     (1) 
     edge [->, blue] node {} (2)

     (2)
     edge node {} (3)
     ;

    \end{tikz_mrfou}
    \begin{tikz_mrfou}

     %% Nodes %%
     \node[bluenode] (0) {$0$};
     \node[bluenode, above right of=0] (1) {$1$};
     \node[bluenode, below right of=1] (2) {$2$};
     \node[bluenode, below right of=0] (3) {$3$};

     %% Edges %%
     \path[-]

     (0)
     edge [->, blue] node {} (1)
     
     (1) 
     edge [->, blue] node {} (2)

     (2)
     edge [->, blue] node {} (3)
     ;

    \end{tikz_mrfou}
    \begin{tikz_mrfou}

     %% Nodes %%
     \node[bluenode] (0) {$0$};
     \node[bluenode, above right of=0] (1) {$1$};
     \node[bluenode, below right of=1] (2) {$2$};
     \node[bluenode, below right of=0] (3) {$3$};

     %% Edges %%
     \path[-]

     (0)
     edge [->, blue] node {} (1)
     
     (1) 
     edge [->, blue] node {} (2)

     (2)
     edge [<-, red] node {$true$} (3)
     ;

    \end{tikz_mrfou}
    \begin{tikz_mrfou}

     %% Nodes %%
     \node[bluenode] (0) {$0$};
     \node[bluenode, above right of=0] (1) {$1$};
     \node[rednode, below right of=1] (2) {$2$};
     \node[bluenode, below right of=0] (3) {$3$};

     %% Edges %%
     \path[-]

     (0)
     edge [->, blue] node {} (1)
     
     (1) 
     edge [<-, red] node {$false$} (2)

     (2)
     edge [<-, red] node {$true$} (3)
     ;

    \end{tikz_mrfou}
    \begin{tikz_mrfou}

     %% Nodes %%
     \node[bluenode] (0) {$0$};
     \node[bluenode, above right of=0] (1) {$1$};
     \node[rednode, below right of=1] (2) {$2$};
     \node[bluenode, below right of=0] (3) {$3$};

     %% Edges %%
     \path[-]

     (0)
     edge [<-, red] node {$true$} (1)
     
     (1) 
     edge [<-, red] node {$false$} (2)

     (2)
     edge [<-, red] node {$true$} (3)
     ;

    \end{tikz_mrfou}
    \begin{tikz_mrfou}

     %% Nodes %%
     \node[rednode] (0) {$0$};
     \node[bluenode, above right of=0] (1) {$1$};
     \node[rednode, below right of=1] (2) {$2$};
     \node[bluenode, below right of=0] (3) {$3$};

     %% Edges %%
     \path[-]

     (0)
     edge [<-, red] node {$true$} (1)
     
     (1) 
     edge [<-, red] node {$false$} (2)

     (2)
     edge [<-, red] node {$true$} (3)
     ;

    \end{tikz_mrfou}
   \end{center}
    L'algorithme nous retourne alors le tableau de booléens $[1,0,1,0]$
    qui correspond à la couverture $\{0,2\}$.


   \subsubsection{Implémentation et complexité}
   \indent Nous avons implémenté cet algorithme dans le fichier
   \emph{Tree.cpp}. Les méthodes sont \emph{coverTree} et
   \emph{coverTree\_aux}.\\
   Il n'utilise qu'un parcours en profondeur et des calculs constants
   (le calcul du nombre de voisins d'un sommet est aussi constant car
   nous avons stocké le degré des n\oe{}uds dans la structure du
   graphe). La complexité est donc celle du parcours en profondeur,
   c'est-à-dire linéaire.