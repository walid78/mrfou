  \subsection{Question 2}
  Le but ici est de faire la comparaison entre l'algorithme du projet vu
  dans la question précédente (\ref{part2q1} page \pageref{part2q1}) et
  l'alogorithme vu en cours basé sur un couplage maximum.\newline
  \indent L'algorithme du couplage maximum est implémenté dans le fichier
  \emph{Graph.cpp} par la méthode \emph{coverCourses()} et l'algorithme
  du projet est implémenté dans \emph{Tree.cpp} par la méthode
  \emph{coverProject()} et la fonction \emph{coverProject\_aux()}.\newline
  \indent Pour comparer ces deux algorithmes, nous allons mesurer leur
  temps d'exécution et surtout la taille de la couverture trouvée. Le
  résultat de ces mesures se trouve dans le tableau \ref{tableau}
  page \pageref{tableau}. Toutes les comparaisons sont faites sur trois
  graphes: \emph{ex32-12.gin, testTree1000.gin, testTree100.gin}.\newline
  \indent Le calcul des temps d'exécution ce fait à l'aide des fichiers
  \emph{timing.cc} et \emph{timing.h} que nous avons récupérés des
  projets \emph{Padico\footnote{http://gforge.inria.fr/projects/padico}}
  et \emph{Adage\footnote{http://gforge.inria.fr/projects/adage}} de
  l'\emph{INRIA.}\newline

  \begin{figure}[!ht]
   \begin{center}
    \begin{tabular}{|c|c|c|c|c|c|c|}
     \cline{2-7}
     \multicolumn{1}{c|}{} & \multicolumn{2}{|c|}{ex32-12.gin}
     &\multicolumn{2}{|c|}{testTree100.gin} &
     \multicolumn{2}{|c|}{testTree1000.gin}\\ 
     \cline{2-7}
     \multicolumn{1}{c|}{} & algoP & algoC & algoP & algoC & algoP &
     algoC\\
     \hline
     Temps d'execution(en $\mu$s) & 0&0 &1 &1 &2& 2\\
     \hline
     Taille de la couverture & 0&0 & 1&1 & 2&2\\
     \hline
    \end{tabular}
    \caption{Tableau comparatif des exécutions des plusieurs
    exemples (algoP: Algorithme du projet et algoC: Algorithme par
    couplage maximum).\label{tableau}} 
   \end{center}
  \end{figure}  