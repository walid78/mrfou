  \subsection{Question 4}
  Montrer que $G$ possède un couplage $C$ de taille $|S|/2$.\\
  
  \emph{Inachevé : Nous avons voulu donc montrer qu'il n'existe aucun
  couplage maximal pour $G$ dont la taille soit strictement plus petite
  que $|S|/2$. C'est-à-dire que pour tout couplage $C$ tel que $|C| <
  |S|/2$, on peut créer un couplage $C'$ qui est $C$ auquel on a ajouté
  une arête de $G$. C'est donc que $C$ n'était pas maximal.}

  \emph{Nous n'avons malheureusement pas réussi à démontrer cette
  propriété, ni par cette dernière méthode (nous avons voulu montrer par
  induction sur le nombre de sommets que $|C| \geq |S|/2$), ni avec
  d'autres idées que nous avons eues (démonstration par induction sur la
  taille de $S$ ou $C$, démonstration par calcul du nombre d'arêtes
  minimum de $C$, \dots). Nous allons donc admettre cette propriété pour
  montrer la $2$-approximation.}\\

  $C$ est un couplage du graphe, donc on sait que toute couverture doit
  être de taille au moins $|C|$ (il est obligatoire de choisir au moins
  l'un des deux sommets de chacune des arêtes du couplage). On a donc :
  \begin{center}
   $|OPT| \geq |C|$ $\iff$ $|OPT| \geq |S|/2$ $\iff$ $2 * |OPT| \geq
   |S|$
  \end{center}

  Nous avons donc $|S| \leq 2 * |OPT|$ et on sait que l'algorithme
  atteint le cas double (voir question précédente \ref{part2q3} page
  \pageref{part2q3}), on peut donc en déduire que l'algorithme du projet
  est $2$-approché.
  