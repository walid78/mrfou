  \subsection{Question 3}
  Considérons le graphe segment de trois sommets suivant :
  \begin{center}
   \begin{tikz_mrfou}

    %% Nodes %%
    \node[bluenode] (0) {$0$};
    \node[bluenode, right of=0] (1) {$1$};
    \node[bluenode, right of=1] (2) {$2$};

    %% Edges %%
    \path[-]

    (0)
    edge node {} (1)
    
    (1) 
    edge node {} (2)
    ;

   \end{tikz_mrfou}
  \end{center}

  La couverture minimale de ce graphe est $\{1\}$ de taille 1.

  \begin{center}
   \begin{tikz_mrfou}
    
    %% Nodes %%
    \node[bluenode] (0) {$0$};
    \node[rednode, right of=0] (1) {$1$};
    \node[bluenode, right of=1] (2) {$2$};

    %% Edges %%
    \path[-]

    (0)
    edge node {} (1)
    
    (1) 
    edge node {} (2)
    ;

   \end{tikz_mrfou}
  \end{center}

  L'algorithme du projet, lui, va retourner les sommets non feuilles de
  ce graphe qui est un arbre, c'est-à-dire par exemple $\{0,1\}$ qui
  sera de taille deux fois plus grande.

  \begin{center}
   \begin{tikz_mrfou}
    
    %% Nodes %%
    \node[rednode] (0) {$0$};
    \node[rednode, right of=0] (1) {$1$};
    \node[bluenode, right of=1] (2) {$2$};

    %% Edges %%
    \path[-]

    (0)
    edge node {} (1)
    
    (1) 
    edge node {} (2)
    ;

   \end{tikz_mrfou}
  \end{center}
