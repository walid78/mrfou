% TO COMPILE %

%%%%%%%%%%                             %%%%%%%%%%
%%%%%%%%%% AUTO INSERTION D'UNE ENTETE %%%%%%%%%%
%%%%%%%%%%                             %%%%%%%%%%
%%%%%%%%%%     POUR UN FICHIER TEX     %%%%%%%%%%
%%%%%%%%%%                             %%%%%%%%%%

%%%%%%          %%%%%%
%%%%%% PACKAGES %%%%%%
%%%%%%          %%%%%%

\documentclass[a4paper]{article}
\usepackage{fullpage}
\usepackage[utf8]{inputenc}
\usepackage[francais]{babel}
\usepackage{graphicx}
\usepackage{listings}
\usepackage{color} 
\usepackage{enumerate}
\usepackage{amsfonts}
\usepackage{amssymb}
\usepackage{amsmath}
\usepackage{wasysym}
\usepackage{chemarrow}
\usepackage[colorlinks=true, urlcolor=blue]{hyperref}
\usepackage[boxed,linesnumbered,ruled]{algorithm2e}
\usepackage{tikz}
\usetikzlibrary{positioning}
\usetikzlibrary{automata}

%%%%%%              %%%%%%
%%%%%% MISE EN PAGE %%%%%%
%%%%%%              %%%%%%


%%%%% Marges %%%%%
%\addtolength{\textwidth}{3cm}
\addtolength{\oddsidemargin}{-1.3cm}
\textheight=24cm % longueur utile de la page
\topmargin=-1cm % marge haute
%\bottommargin=-1cm % marge haute
\headsep=20pt % séparation 20 points entre entête et texte
%\footskip=30pt % idem séparation pied de page
\addtolength{\footskip}{20pt}

%%%%% Entetes et Enpieds de pages %%%%%
\usepackage{fancyhdr}
\pagestyle{fancy}
\lhead{\footnotesize{\textit{Ludovic Brochard, Fabien Kuntz, Benoît Védrenne}}}
\rhead{\footnotesize{\textit{Projet Algorithmes du Monde Réel}}}

%%%%% Numérotation sections %%%%
\setcounter{secnumdepth}{3}
\setcounter{tocdepth}{3}

%%%%% Types enumerate %%%%%
\let\enumeratesav\enumerate
\let\endenumeratesav\endenumerate

\renewenvironment{enumerate}{
\enumeratesav
 \setlength{\itemsep}{0.5pt}
 \setlength{\parskip}{0pt}
 \setlength{\parsep}{0pt}}
\endenumeratesav

\renewcommand{\labelenumi}{-}
\renewcommand{\labelenumii}{$\bullet$}
\renewcommand{\labelenumiii}{$\star$}
\renewcommand{\labelenumiv}{\_}

\newenvironment{tikz_mrfou}{
\begin{tikzpicture}[>=stealth,shorten >=1pt,auto,node
 distance=2cm,on grid,semithick,inner sep=2pt,bend
 angle=30,font=\scriptsize, initial text=]
}{\end{tikzpicture}}


%%%%% Titre %%%%%
\newlength{\larg}
\setlength{\larg}{14.5cm}

%%%Zone de Code%%%
\definecolor{gray}{gray}{0.86}  

\lstset{numbers=left, tabsize=2, frame=single, breaklines=true,
basicstyle=\ttfamily,numberstyle=\tiny\ttfamily, framexleftmargin=13mm,
backgroundcolor=\color{gray}, xleftmargin=14mm} 

\newsavebox{\fmbox}
\newenvironment{fmpage}[1]{\begin{lrbox}{\fmbox}\begin{minipage}{#1}}{\end{minipage}\end{lrbox}\fbox{\usebox{\fmbox}}}


%%%%%%          %%%%%%
%%%%%% DOCUMENT %%%%%%
%%%%%%          %%%%%%


\begin{document}


%%%%% Page de présentation %%%%%
\thispagestyle{empty}

\setlength{\unitlength}{1in}

%%%%% Image de fond %%%%%
% \begin{picture}(0,0)
% \put(0.2,-7){\includegraphics[width=\textwidth]{images/bx1_trans.eps}}
% \end{picture}

\begin{flushright}
 \noindent {\rule{\larg}{0.5mm}}
\end{flushright}
\vspace{7mm}
\begin{flushright}
 \Huge{\bf Algorithmes du Monde Réel} \\
 \Huge{\bf Projet} \\
 ~\\
 \huge{Partie 2 - Approximations}\\
 % \Large{\bf{\emph{Fabien Kuntz}}}
\end{flushright}
\vspace{7mm}
\begin{flushright}
 {\rule{\larg}{0.5mm}}
\end{flushright}
\vspace{2mm}
\begin{flushright}
 \large{\bf Professeurs : M. Zeitoun, A. Muscholl, C. Gavoille} \\
 ~\\
 \large{Master 2 Informatique}\\
 ~\\
 % \today
 \vspace{10cm}
 \large{Ludovic Brochard, Fabien Kuntz, Benoît Védrenne}
{\rule{\larg}{0.5mm}}
\end{flushright}

\newpage

\addtolength{\oddsidemargin}{1cm}

%%%%% Table des matières %%%%%
\thispagestyle{empty}
\tableofcontents
\newpage

\setcounter{page}{1}


%%%%%%              %%%%%%
%%%%%% ZONE DE CODE %%%%%%
%%%%%%              %%%%%%

   \subsection{Question 1}
  Soit $F$ une forêt et $uv$ une arête telle que le degré de $v$ est
  $1$.\\

  Considérons un couverture $U$ de $F$ qui ne contient pas $u$. La
  couverture ne contenant pas $u$ et $uv$ étant une arête, alors $U$
  contient nécessairement $v$.

  Or, le degré de $v$ est $1$, donc le fait que $v$ soit dans $U$ ne
  sert qu'à couvrir l'arête $uv$. On peut donc choisir $u$ dans $U$ à la
  place de $v$, la seule arête que $v$ couvrait est alors aussi couverte
  par $u$ et la taille de la couverture est la même.
  
  On peut donc construire à partir d'une couverture minimale $U$ de $F$
  qui ne contient pas $u$, la couverture $U'$ de même taille qui
  contient $u$.\\

  
   \subsection{Question 2}
  
    \paragraph{Idée de l'algorithme :\\}
    Afin de calculer une couverture de taille minimale sur un arbre $T$,
    nous utilisons un parcours en profondeur de l'arbre.\\
    En utilisant la Question 1 (\ref{part1q1} page \pageref{part1q1}),
    nous marquons les sommets qui sont parents d'une ou plusieurs
    feuilles car les feuilles sont de degré $1$. Ensuite, lors de la
    partie ``retour'' de la récursion, nous allons marqué les sommets
    qui sont pères de sommets non marqués.\\

    Pour qu'un sommet sache s'il doit être marqué, chacun de ses fils va
    lui dire \emph{''Marque toi !''}($true$) ou \emph{''Je suis déjà
    marqué donc tu n'as pas besoin de l'être vis-à-vis de
    moi.''}($false$).

    Marquer un sommet revient à mettre la case qui lui correspond dans
    le tableau de booléen $cover$ à $true$.

    \paragraph{Algorithme :\\}
    \begin{algorithm}
     \SetLine
     $\mathbf{global~var}$ $cover$ : Boolean table;\\
     \Begin{
       \ForEach{$v \in V$}{
         $cover[v] := false$;
       }
       coverTree\_aux($root$);\\
       ~\\
       \Return{$cover$};
     }
     \caption{coverTree(): Boolean table}
    \end{algorithm}

    \begin{algorithm}
     \SetLine
     $\mathbf{local~var}$ $markFather := true$ : Boolean;\\
     \Begin{
       \If{$v$ is a leaf}{
         \Return{$true$};
       }
       \ForEach{$(v,w) \in E$}{
         \If{coverTree\_aux($w$)}{
           $cover[node] := true$;\\
           $markFather := false$;
         }
       }
       ~\\
       \Return{$markFather$};
     }
     \caption{coverTree\_aux(Vertex $v$): Booléen}
    \end{algorithm}
     
    \paragraph{Exemple :\\}
    \tikzstyle{bluenode}=[circle,draw=blue!50,fill=blue!20,thick, inner
    sep=0pt,minimum size=6mm]

    \tikzstyle{rednode}=[circle,draw=red!50,fill=red!20,thick, inner
    sep=0pt,minimum size=6mm]

%     \tikzstyle{oddnode}=[rectangle,draw=black!50,fill=black!20,thick, inner
%     sep=0pt,minimum size=5mm]  
    
    \begin{center}
     \begin{tikz_mrfou}

      %% Nodes %%
      \node[bluenode] (0) {$0$};
      \node[bluenode, above right of=0] (1) {$1$};
      \node[bluenode, below right of=1] (2) {$2$};
      \node[bluenode, below right of=0] (3) {$3$};

      %% Edges %%
      \path[-]

      (0)
      edge node {} (1)
      edge node {} (2)
      edge node {} (3)
      
      (1) 
      edge node {} (2)

      (2)
      edge node {} (3)
      ;

     \end{tikz_mrfou}
    \end{center}

    À partir du graphe, on calcule un arbre correspondant par un
    parcours en profondeur à partir du sommet $0$ (à l'aide du
    constructeur de \emph{Tree.cpp}).

    \begin{center}
     \begin{tikz_mrfou}

      %% Nodes %%
      \node[bluenode] (0) {$0$};
      \node[bluenode, above right of=0] (1) {$1$};
      \node[bluenode, below right of=1] (2) {$2$};
      \node[bluenode, below right of=0] (3) {$3$};

      %% Edges %%
      \path[-]

      (0)
      edge node {} (1)
      
      (1) 
      edge node {} (2)

      (2)
      edge node {} (3)
      ;

     \end{tikz_mrfou}
    \end{center}

    On applique alors notre algorithme.

    \begin{center}
     \begin{tikz_mrfou}

      %% Nodes %%
      \node[bluenode] (0) {$0$};
      \node[bluenode, above right of=0] (1) {$1$};
      \node[bluenode, below right of=1] (2) {$2$};
      \node[bluenode, below right of=0] (3) {$3$};

      %% Edges %%
      \path[-]

      (0)
      edge [->, blue] node {} (1)
      
      (1) 
      edge node {} (2)

      (2)
      edge node {} (3)
      ;

     \end{tikz_mrfou}
     \begin{tikz_mrfou}

      %% Nodes %%
      \node[bluenode] (0) {$0$};
      \node[bluenode, above right of=0] (1) {$1$};
      \node[bluenode, below right of=1] (2) {$2$};
      \node[bluenode, below right of=0] (3) {$3$};

      %% Edges %%
      \path[-]

      (0)
      edge [->, blue] node {} (1)
      
      (1) 
      edge [->, blue] node {} (2)

      (2)
      edge node {} (3)
      ;

     \end{tikz_mrfou}
     \begin{tikz_mrfou}

      %% Nodes %%
      \node[bluenode] (0) {$0$};
      \node[bluenode, above right of=0] (1) {$1$};
      \node[bluenode, below right of=1] (2) {$2$};
      \node[bluenode, below right of=0] (3) {$3$};

      %% Edges %%
      \path[-]

      (0)
      edge [->, blue] node {} (1)
      
      (1) 
      edge [->, blue] node {} (2)

      (2)
      edge [->, blue] node {} (3)
      ;

     \end{tikz_mrfou}
     \begin{tikz_mrfou}

      %% Nodes %%
      \node[bluenode] (0) {$0$};
      \node[bluenode, above right of=0] (1) {$1$};
      \node[bluenode, below right of=1] (2) {$2$};
      \node[bluenode, below right of=0] (3) {$3$};

      %% Edges %%
      \path[-]

      (0)
      edge [->, blue] node {} (1)
      
      (1) 
      edge [->, blue] node {} (2)

      (2)
      edge [<-, red] node {$true$} (3)
      ;

     \end{tikz_mrfou}
     \begin{tikz_mrfou}

      %% Nodes %%
      \node[bluenode] (0) {$0$};
      \node[bluenode, above right of=0] (1) {$1$};
      \node[rednode, below right of=1] (2) {$2$};
      \node[bluenode, below right of=0] (3) {$3$};

      %% Edges %%
      \path[-]

      (0)
      edge [->, blue] node {} (1)
      
      (1) 
      edge [<-, red] node {$false$} (2)

      (2)
      edge [<-, red] node {$true$} (3)
      ;

     \end{tikz_mrfou}
     \begin{tikz_mrfou}

      %% Nodes %%
      \node[bluenode] (0) {$0$};
      \node[bluenode, above right of=0] (1) {$1$};
      \node[rednode, below right of=1] (2) {$2$};
      \node[bluenode, below right of=0] (3) {$3$};

      %% Edges %%
      \path[-]

      (0)
      edge [<-, red] node {$true$} (1)
      
      (1) 
      edge [<-, red] node {$false$} (2)

      (2)
      edge [<-, red] node {$true$} (3)
      ;

     \end{tikz_mrfou}
     \begin{tikz_mrfou}

      %% Nodes %%
      \node[rednode] (0) {$0$};
      \node[bluenode, above right of=0] (1) {$1$};
      \node[rednode, below right of=1] (2) {$2$};
      \node[bluenode, below right of=0] (3) {$3$};

      %% Edges %%
      \path[-]

      (0)
      edge [<-, red] node {$true$} (1)
      
      (1) 
      edge [<-, red] node {$false$} (2)

      (2)
      edge [<-, red] node {$true$} (3)
      ;

     \end{tikz_mrfou}
    \end{center}
     L'algorithme nous retourne alors la couverture : $[1,0,1,0]$.


    \paragraph{Implémentation et complexité :\\}

    \indent Nous avons implémenté cet algorithme dans le fichier
    \emph{Tree.cpp}. Les méthodes sont \emph{coverTree} et
    \emph{coverTree\_aux}.

    Notre algorithme n'utilise qu'un parcours en profondeur et des
    calculs constants (le calcul du nombre de voisins d'un sommet est
    aussi constant car nous avons stocké le degré d'un n\oe{}ud dans la
    structure de graphe), il est donc de la complexité du parcours en
    profondeur, c'est-à-dire linéaire.
 
  \section{VC et graphes}
	
  \subsection{Question 1\label{part2q1} }

   \subsection{Question 2}
  Le but, ici, est de faire la comparaison entre l'algorithme vu dans la
  partie 2.1 et l'alogorithme basé sur un couplage maximum.

 \subsection{Question 3}

   \subsection{Question 4}
  Montrer que $G$ possède un couplage $C$ de taille $|S|/2$.\\
  
  \emph{Inachevé : Nous avons voulu donc montrer qu'il n'existe aucun
  couplage maximal pour $G$ dont la taille soit strictement plus petite
  que $|S|/2$. C'est-à-dire que pour tout couplage $C$ tel que $|C| <
  |S|/2$, on peut créer un couplage $C'$ qui est $C$ où on a ajouté une
  arête de $G$. C'est donc que $C$ n'était pas maximal.}

  \emph{Nous n'avons malheureusement pas réussi à démontrer cette
  propriété, ni par cette dernière méthode, ni avec d'autres idées que
  nous avons eues (démonstration par induction sur la taille de $S$ ou
  $C$, démonstration par calcul du nombre d'arêtes minimum de $C$,
  \dots). Nous allons donc admettre cette propriété pour montrer la
  $2$-approximation.}\\

  $C$ est un couplage du graphe, donc on sait que toute couverture doit
  être de taille au moins $|C|$ (il est obligatoire de choisir au moins
  l'un des deux sommets de chacune des arêtes du couplage). On a donc :
  \begin{center}
   $|OPT| \geq |C|$ $\iff$ $|OPT| \geq |S|/2$ $\iff$ $2 * |OPT| \geq
   |S|$
  \end{center}

  Nous avons donc $|S| \leq 2 * |OPT|$ et on sait que l'algorithme
  atteint le cas double (voir question précédente \ref{part2q3} page
  \pageref{part2q3}), on peut donc en déduire que l'algorithme du projet
  est $2$-approché.
  

%%%%%%               %%%%%%
%%%%%% /ZONE DE CODE %%%%%%
%%%%%%               %%%%%%

\end{document}

%%%%%%           %%%%%%
%%%%%% /DOCUMENT %%%%%%
%%%%%%           %%%%%%




%%%%% Texte italique %%%%%
%  \textit{} 

%%%%% Liste %%%%
%%% Itemize %%%
%  \begin{itemize}
%   \item item1\\
%   \item item2
%  \end{itemize}

%%% Enumerate %%%
%  \begin{enumerate}
%   \item item1
%   \item item2
%  \end{enumerate}

%%%%% Tabulations %%%%%
%   \begin{tabbing}
%    XX\=XX\=\kill
%    \>(OrdresEnonce.v, ligne 244)\\
%    \>\>test2
%   \end{tabbing}
 
%%%%% Note de pied de page %%%%%
%  \footnote{test}

%%%%% Référence %%%%%
%   \label{ref}
%% Plus loin :
%   \ref{ref}
%   \pageref{ref}

%%%%% Code %%%%%
% \begin{lstlisting}
%      List<Integer> lexBFS2 = new ArrayList<Integer>();
%      lexBFS2.add(3);
%      lexBFS2.add(2);
%      lexBFS2.add(4);
%      lexBFS2.add(1);
%      lexBFS2.add(0);    
%      assertNotNull(lexBFS2);	
%      assertEquals(lexBFS2,Graphs.lexBFS(nogYComp));
%   \end{lstlisting}

%%%%% Figure %%%%%
%  \begin{figure}[!ht]
%   \begin{center}
%	\includegraphics[width=7cm]{figs/cours1/fig2.eps}
%	\caption{\emph{MT2} : Calcul non déterministe}
%   \end{center}
%  \end{figure}
