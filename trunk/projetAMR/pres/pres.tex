%%%%% Remove ? to make this file compilable %%%%%
% TO COMPILE %

%%%%%%%%%%                             %%%%%%%%%%
%%%%%%%%%% AUTO INSERTION D'UNE ENTETE %%%%%%%%%%
%%%%%%%%%%                             %%%%%%%%%%
%%%%%%%%%%     POUR UN FICHIER TEX     %%%%%%%%%%
%%%%%%%%%%                             %%%%%%%%%%

%%%%%%          %%%%%%
%%%%%% PACKAGES %%%%%%
%%%%%%          %%%%%%

\documentclass[8pt]{beamer}
\usepackage[utf8]{inputenc}
\usepackage[french]{babel}
\usepackage[babel=true,kerning=true]{microtype} % probleme avec : tikz
\usepackage{graphicx}
\usepackage{listings}
\usepackage{color} 
\usepackage{enumerate}
\usepackage{amsfonts}
\usepackage{amssymb}
\usepackage{amsmath}
\usepackage{wasysym}
\usepackage{chemarrow}
\usepackage{tikz}
\usetikzlibrary{positioning}
\usetikzlibrary{automata}
\usetikzlibrary{trees}
\usetikzlibrary{arrows}
\usepackage{beamerthemesplit}
\usetheme{JuanLesPins}



%%%%%%              %%%%%%
%%%%%% MISE EN PAGE %%%%%%
%%%%%%              %%%%%%


%%%%% Environnements %%%%%

\newenvironment{tikz_mrfou}{
\begin{tikzpicture}[>=stealth,shorten >=1pt,auto,node
 distance=1.5cm,on grid, semithick, minimum size=5mm,bend
 angle=10,font=\small, initial text=, label distance=-1mm]
}{\end{tikzpicture}}

\definecolor{turq}{rgb}{0,0.60,0.60}

\tikzstyle{evennode}=[circle,draw=blue!50,fill=blue!20,thick, inner
sep=0pt,minimum size=6mm]

\tikzstyle{oddnode}=[rectangle,draw=black!50,fill=black!20,thick, inner
sep=0pt,minimum size=5mm]

\tikzstyle{pointnode}=[circle,draw=black,fill=black, thick, inner
sep=0pt,minimum size=1mm] 

\tikzstyle{gnode}=[circle, draw=blue!50, thick, inner sep=0pt,
minimum size=5mm]

\tikzstyle{bluenode}=[circle, draw=blue!50, fill=blue!20, thick, inner
sep=0pt, minimum size=5mm]

\tikzstyle{rednode}=[circle, draw=red!50, fill=red!20, thick, inner
sep=0pt, minimum size=5mm]

\tikzstyle{greennode}=[circle, draw=green!50, fill=green!20, thick, inner
sep=0pt, minimum size=5mm]

\tikzstyle{blacknode}=[circle, draw=black!50, fill=black!20, thick, inner
sep=0pt, minimum size=5mm]

\newcommand{\Cb}[1]{\textcolor{black}{#1}}

%%%%%%          %%%%%%
%%%%%% DOCUMENT %%%%%%
%%%%%%          %%%%%%


\begin{document}

%%%%%%              %%%%%%
%%%%%% ZONE DE CODE %%%%%%
%%%%%%              %%%%%%

%% Style bloc %%
\setbeamertemplate{blocks}[rounded][shadow=true]
\setbeamercovered{dynamic}

%% Page de garde %%
\author[Ludovic Brochard, Fabien Kuntz, Benoît Védrenne]{Ludovic
Brochard, Fabien Kuntz, Benoît Védrenne\\ 
~\\
\small Professeurs : Cyril Gavoille, Anca Muscholl, Marc Zeitoun,
Alexander Zvonkin}
\institute{\footnotesize Module AMR\\~\\Master S\&T
Informatique\\~\\Université Bordeaux I\\} 
\title{Soutenances d'\textit{Algorithmes du Monde Réel}} 
\date{22 janvier 2009 - 11h40}

\setcounter{page}{1}

%% Debut %%
\frame{\titlepage}
\frame{\tableofcontents}

 \section{Réductions}
 %==============================================================================
 \frame{\tableofcontents[current]}
 %% reduc.tex %%
   \subsection{Instance positive}
   \begin{frame}
    \frametitle{4-Col}

    \begin{block}{\only<1,2>{Données}\only<3->{Solution}}
     \begin{center}
      \begin{figure}[!ht]
       \only<1,2>{
       \begin{tikz_mrfou}
	%% Nodes %%
	\node[gnode] (0) {0};
	\node[gnode, below left of=0] (1) {1};
	\node[gnode, below right of=1] (2) {2};
	\node[gnode, below right of=0] (3) {3};
	\node[gnode, above left of=1] (4) {4};
	\node[gnode, below left of=1] (5) {5};

	%% Edges %%
	\path[-] 
	(0)
	edge node {} (1)
	edge node {} (2)
	edge node {} (3)

	(1) 
	edge node {} (2)
	edge node {} (3)
	edge node {} (4)
	edge node {} (5)

	(2)
	edge node {} (3)

	;
	
       \end{tikz_mrfou}
       \caption{Graphe \emph{poisson}}
       }\only<3->{
       \begin{tikz_mrfou}
	%% Nodes %%
	\node[bluenode] (0) {0};
	\node[rednode, below left of=0] (1) {1};
	\node[greennode, below right of=1] (2) {2};
	\node[blacknode, below right of=0] (3) {3};
	\node[blacknode, above left of=1] (4) {4};
	\node[blacknode, below left of=1] (5) {5};

	%% Edges %%
	\path[-] 
	(0)
	edge node {} (1)
	edge node {} (2)
	edge node {} (3)

	(1) 
	edge node {} (2)
	edge node {} (3)
	edge node {} (4)
	edge node {} (5)

	(2)
	edge node {} (3)

	;
	
       \end{tikz_mrfou}
       \caption{Graphe \emph{poisson} 4-colorié}
       }
      \end{figure}
     \end{center}
    \end{block}

   \uncover<2->{
   \begin{block}{Question\only<3->{/Réponse}}
    Le graphe est-il 4-coloriable ? \only<3->{\textcolor{turq}{Oui ! \Large\smiley}}
   \end{block}
   }
   \end{frame}

  \subsection{Instance négative}
  \begin{frame}
   \frametitle{3-Col}

   \begin{block}{Données}
    \begin{center}
     \begin{figure}[!ht]
      \begin{tikz_mrfou}
       %% Nodes %%
       \node[gnode] (0) {0};
       \node[gnode, below left of=0] (1) {1};
       \node[gnode, below right of=1] (2) {2};
       \node[gnode, below right of=0] (3) {3};
       \node[gnode, above left of=1] (4) {4};
       \node[gnode, below left of=1] (5) {5};

       %% Edges %%
       \path[-] 
       (0)
       edge node {} (1)
       edge node {} (2)
       edge node {} (3)

       (1) 
       edge node {} (2)
       edge node {} (3)
       edge node {} (4)
       edge node {} (5)

       (2)
       edge node {} (3)

       ;
       
      \end{tikz_mrfou}
      \caption{Graphe \emph{poisson}}
     \end{figure}
    \end{center}
   \end{block}

   \uncover<2->{
   \begin{block}{Question\only<3->{/Réponse}}
    Le graphe est-il 3-coloriable ? \only<3->{\textcolor{red}{Non !
    \Large \frownie}}
   \end{block}
   }
  \end{frame}

  \subsection{Bonus: les ``cas faciles''}
   

 \section{Couverture par sommets}
 %==============================================================================
 \frame{\tableofcontents[current]}
 %% vc.tex %%
  \subsection{Présentation de l'algorithme}
  \begin{frame}
   \frametitle{Algorithme}
   \begin{block}{Idée}
    \begin{itemize}
     \item Méthode: Parcours en profondeur
     \item Aller: Marquer les pères d'au moins une feuille
     \item Retour: Marquer les sommets pères de sommets non marqués
    \end{itemize}
   \end{block}

   \begin{block}{Complexité}
    La complexité est celle du parcours en profondeur : $O(|V| + |E|)$.
   \end{block}
  \end{frame}

  \subsection{Exemple d'exécution}
  \begin{frame}
   \frametitle{Exécution}

   \begin{block}{\only<-19>{Application de l'algorithme}\only<20->{Solution}}
    \begin{center}
     \begin{figure}[!ht]
      \only<1>{
      \begin{tikzpicture}
       \tikzstyle{level 1}=[level distance=1.2cm, sibling distance=1.2cm]
       \tikzstyle{childnode}=[rounded corners]
       \tikzstyle{marknode}=[fill=red!30, rounded corners]
       \tikzstyle{edge from parent}=[draw,thick]
       \node% [fill=red!30]
       {0} 
        child {node[childnode] {1} 
         child {node[childnode] {2}
          child {node[childnode] {3}
          }
         }
         child {node[childnode] {4}
          child {node[childnode] {5}
          }
          child {node[childnode] {6} 
           child {node[childnode] {7}}
          }
         }
        };
      \end{tikzpicture}
      }\only<2>{
      \begin{tikzpicture}
       \tikzstyle{level 1}=[level distance=1.2cm, sibling distance=1.2cm]
       \tikzstyle{childnode}=[rounded corners, black]
       \tikzstyle{marknode}=[fill=red!30, rounded corners]
       \tikzstyle{edge from parent}=[draw,thick]
       \node {0} 
        child[blue, ->] {node[childnode] {1} 
         child[black, -] {node[childnode] {2}
          child {node[childnode] {3}
          }
         }
         child[black, -] {node[childnode] {4}
          child {node[childnode] {5}
          }
          child {node[childnode] {6} 
           child {node[childnode] {7}}
          }
         }
        };
      \end{tikzpicture}
      }\only<3>{
      \begin{tikzpicture}
       \tikzstyle{level 1}=[level distance=1.2cm, sibling distance=1.2cm]
       \tikzstyle{childnode}=[rounded corners, black]
       \tikzstyle{marknode}=[fill=red!30, rounded corners]
       \tikzstyle{edge from parent}=[draw,thick]
       \node {0} 
        child[blue, ->] {node[childnode] {1} 
         child {node[childnode] {2}
          child[black, -] {node[childnode] {3}
          }
         }
         child[black, -] {node[childnode] {4}
          child {node[childnode] {5}
          }
          child {node[childnode] {6} 
           child {node[childnode] {7}}
          }
         }
        };
      \end{tikzpicture}
      }\only<4>{
      \begin{tikzpicture}
       \tikzstyle{level 1}=[level distance=1.2cm, sibling distance=1.2cm]
       \tikzstyle{childnode}=[rounded corners, black]
       \tikzstyle{marknode}=[fill=red!30, rounded corners]
       \tikzstyle{edge from parent}=[draw,thick]
       \node {0} 
        child[blue, ->] {node[childnode] {1} 
         child {node[childnode] {2}
          child {node[childnode] {3}
          }
         }
         child[black, -] {node[childnode] {4}
          child {node[childnode] {5}
          }
          child {node[childnode] {6} 
           child {node[childnode] {7}}
          }
         }
        };
      \end{tikzpicture}
      }\only<5>{
      \begin{tikzpicture}
       \tikzstyle{level 1}=[level distance=1.2cm, sibling distance=1.2cm]
       \tikzstyle{childnode}=[rounded corners, black]
       \tikzstyle{marknode}=[fill=red!30, rounded corners]
       \tikzstyle{edge from parent}=[draw,thick]
       \node {0} 
        child[blue, ->] {node[childnode] {1} 
         child {node[childnode] {2}
          child[red, <-] {node[childnode] {3}
           edge from parent node[left] {true}
          }
         }
         child[black, -] {node[childnode] {4}
          child {node[childnode] {5}
          }
          child {node[childnode] {6} 
           child {node[childnode] {7}}
          }
         }
        };
      \end{tikzpicture}
      }\only<6>{
      \begin{tikzpicture}
       \tikzstyle{level 1}=[level distance=1.2cm, sibling distance=1.2cm]
       \tikzstyle{childnode}=[rounded corners, black]
       \tikzstyle{marknode}=[fill=red!30, rounded corners]
       \tikzstyle{edge from parent}=[draw,thick]
       \node {0} 
        child[blue, ->] {node[childnode] {1} 
         child {node[marknode] {\Cb{2}}
          child[red, <-] {node[childnode] {3}
           edge from parent node[left] {true}
          }
         }
         child[black, -] {node[childnode] {4}
          child {node[childnode] {5}
          }
          child {node[childnode] {6} 
           child {node[childnode] {7}}
          }
         }
        };
      \end{tikzpicture}
      }\only<7>{
      \begin{tikzpicture}
       \tikzstyle{level 1}=[level distance=1.2cm, sibling distance=1.2cm]
       \tikzstyle{childnode}=[rounded corners, black]
       \tikzstyle{marknode}=[fill=red!30, rounded corners]
       \tikzstyle{edge from parent}=[draw,thick]
       \node {0} 
        child[blue, ->] {node[childnode] {1} 
         child[red, <-] {node[marknode] {\Cb{2}}
          child {node[childnode] {3}
           edge from parent node[left] {true}
          }
          edge from parent node[left] {false}
         }
         child[black, -] {node[childnode] {4}
          child {node[childnode] {5}
          }
          child {node[childnode] {6} 
           child {node[childnode] {7}}
          }
         }
        };
      \end{tikzpicture}
      }\only<8>{
      \begin{tikzpicture}
       \tikzstyle{level 1}=[level distance=1.2cm, sibling distance=1.2cm]
       \tikzstyle{childnode}=[rounded corners, black]
       \tikzstyle{marknode}=[fill=red!30, rounded corners]
       \tikzstyle{edge from parent}=[draw,thick]
       \node {0} 
        child[blue, ->] {node[childnode] {1} 
         child[red, <-] {node[marknode] {\Cb{2}}
          child {node[childnode] {3}
           edge from parent node[left] {true}
          }
          edge from parent node[left] {false}
         }
         child {node[childnode] {4}
          child[black, -] {node[childnode] {5}
          }
          child[black, -] {node[childnode] {6} 
           child {node[childnode] {7}}
          }
         }
        };
      \end{tikzpicture}
      }\only<9>{
      \begin{tikzpicture}
       \tikzstyle{level 1}=[level distance=1.2cm, sibling distance=1.2cm]
       \tikzstyle{childnode}=[rounded corners, black]
       \tikzstyle{marknode}=[fill=red!30, rounded corners]
       \tikzstyle{edge from parent}=[draw,thick]
       \node {0} 
        child[blue, ->] {node[childnode] {1} 
         child[red, <-] {node[marknode] {\Cb{2}}
          child {node[childnode] {3}
           edge from parent node[left] {true}
          }
          edge from parent node[left] {false}
         }
         child {node[childnode] {4}
          child {node[childnode] {5}
          }
          child[black, -] {node[childnode] {6} 
           child {node[childnode] {7}}
          }
         }
        };
      \end{tikzpicture}
      }\only<10>{
      \begin{tikzpicture}
       \tikzstyle{level 1}=[level distance=1.2cm, sibling distance=1.2cm]
       \tikzstyle{childnode}=[rounded corners, black]
       \tikzstyle{marknode}=[fill=red!30, rounded corners]
       \tikzstyle{edge from parent}=[draw,thick]
       \node {0} 
        child[blue, ->] {node[childnode] {1} 
         child[red, <-] {node[marknode] {\Cb{2}}
          child {node[childnode] {3}
           edge from parent node[left] {true}
          }
          edge from parent node[left] {false}
         }
         child {node[childnode] {4}
          child[red, <-] {node[childnode] {5}
           edge from parent node[left] {true}
          }
          child[black, -] {node[childnode] {6} 
           child {node[childnode] {7}}
          }
         }
        };
      \end{tikzpicture}
      }\only<11>{
      \begin{tikzpicture}
       \tikzstyle{level 1}=[level distance=1.2cm, sibling distance=1.2cm]
       \tikzstyle{childnode}=[rounded corners, black]
       \tikzstyle{marknode}=[fill=red!30, rounded corners]
       \tikzstyle{edge from parent}=[draw,thick]
       \node {0} 
        child[blue, ->] {node[childnode] {1} 
         child[red, <-] {node[marknode] {\Cb{2}}
          child {node[childnode] {3}
           edge from parent node[left] {true}
          }
          edge from parent node[left] {false}
         }
         child {node[marknode] {\Cb{4}}
          child[red, <-] {node[childnode] {5}
           edge from parent node[left] {true}
          }
          child[black, -] {node[childnode] {6} 
           child {node[childnode] {7}}
          }
         }
        };
      \end{tikzpicture}
      }\only<12>{
      \begin{tikzpicture}
       \tikzstyle{level 1}=[level distance=1.2cm, sibling distance=1.2cm]
       \tikzstyle{childnode}=[rounded corners, black]
       \tikzstyle{marknode}=[fill=red!30, rounded corners]
       \tikzstyle{edge from parent}=[draw,thick]
       \node {0} 
        child[blue, ->] {node[childnode] {1} 
         child[red, <-] {node[marknode] {\Cb{2}}
          child {node[childnode] {3}
           edge from parent node[left] {true}
          }
          edge from parent node[left] {false}
         }
         child {node[marknode] {\Cb{4}}
          child[red, <-] {node[childnode] {5}
           edge from parent node[left] {true}
          }
          child {node[childnode] {6} 
           child[black, -] {node[childnode] {7}}
          }
         }
        };
      \end{tikzpicture}
      }\only<13>{
      \begin{tikzpicture}
       \tikzstyle{level 1}=[level distance=1.2cm, sibling distance=1.2cm]
       \tikzstyle{childnode}=[rounded corners, black]
       \tikzstyle{marknode}=[fill=red!30, rounded corners]
       \tikzstyle{edge from parent}=[draw,thick]
       \node {0} 
        child[blue, ->] {node[childnode] {1} 
         child[red, <-] {node[marknode] {\Cb{2}}
          child {node[childnode] {3}
           edge from parent node[left] {true}
          }
          edge from parent node[left] {false}
         }
         child {node[marknode] {\Cb{4}}
          child[red, <-] {node[childnode] {5}
           edge from parent node[left] {true}
          }
          child {node[childnode] {6} 
           child {node[childnode] {7}}
          }
         }
        };
      \end{tikzpicture}
      }\only<14>{
      \begin{tikzpicture}
       \tikzstyle{level 1}=[level distance=1.2cm, sibling distance=1.2cm]
       \tikzstyle{childnode}=[rounded corners, black]
       \tikzstyle{marknode}=[fill=red!30, rounded corners]
       \tikzstyle{edge from parent}=[draw,thick]
       \node {0} 
        child[blue, ->] {node[childnode] {1} 
         child[red, <-] {node[marknode] {\Cb{2}}
          child {node[childnode] {3}
           edge from parent node[left] {true}
          }
          edge from parent node[left] {false}
         }
         child {node[marknode] {\Cb{4}}
          child[red, <-] {node[childnode] {5}
           edge from parent node[left] {true}
          }
          child {node[childnode] {6} 
           child[red, <-] {node[childnode] {7}
            edge from parent node[left] {true}
           }
          }
         }
        };
      \end{tikzpicture}
      }\only<15>{
      \begin{tikzpicture}
       \tikzstyle{level 1}=[level distance=1.2cm, sibling distance=1.2cm]
       \tikzstyle{childnode}=[rounded corners, black]
       \tikzstyle{marknode}=[fill=red!30, rounded corners]
       \tikzstyle{edge from parent}=[draw,thick]
       \node {0} 
        child[blue, ->] {node[childnode] {1} 
         child[red, <-] {node[marknode] {\Cb{2}}
          child {node[childnode] {3}
           edge from parent node[left] {true}
          }
          edge from parent node[left] {false}
         }
         child {node[marknode] {\Cb{4}}
          child[red, <-] {node[childnode] {5}
           edge from parent node[left] {true}
          }
          child {node[marknode] {\Cb{6}}
           child[red, <-] {node[childnode] {7}
            edge from parent node[left] {true}
           }
          }
         }
        };
      \end{tikzpicture}
      }\only<16>{
      \begin{tikzpicture}
       \tikzstyle{level 1}=[level distance=1.2cm, sibling distance=1.2cm]
       \tikzstyle{childnode}=[rounded corners, black]
       \tikzstyle{marknode}=[fill=red!30, rounded corners]
       \tikzstyle{edge from parent}=[draw,thick]
       \node {0} 
        child[blue, ->] {node[childnode] {1} 
         child[red, <-] {node[marknode] {\Cb{2}}
          child {node[childnode] {3}
           edge from parent node[left] {true}
          }
          edge from parent node[left] {false}
         }
         child {node[marknode] {\Cb{4}}
          child[red, <-] {node[childnode] {5}
           edge from parent node[left] {true}
          }
          child[red, <-] {node[marknode] {\Cb{6}} 
           child {node[childnode] {7}
            edge from parent node[left] {true}
           }
           edge from parent node[right] {false}
          }
         }
        };
      \end{tikzpicture}
      }\only<17>{
      \begin{tikzpicture}
       \tikzstyle{level 1}=[level distance=1.2cm, sibling distance=1.2cm]
       \tikzstyle{childnode}=[rounded corners, black]
       \tikzstyle{marknode}=[fill=red!30, rounded corners]
       \tikzstyle{edge from parent}=[draw,thick]
       \node {0} 
        child[blue, ->] {node[childnode] {1} 
         child[red, <-] {node[marknode] {\Cb{2}}
          child {node[childnode] {3}
           edge from parent node[left] {true}
          }
          edge from parent node[left] {false}
         }
         child[red, <-] {node[marknode] {\Cb{4}}
          child {node[childnode] {5}
           edge from parent node[left] {true}
          }
          child {node[marknode] {\Cb{6}} 
           child {node[childnode] {7}
            edge from parent node[left] {true}
           }
           edge from parent node[right] {false}
          }
          edge from parent node[right] {false}
         }
        };
      \end{tikzpicture}
      }\only<18>{
      \begin{tikzpicture}
       \tikzstyle{level 1}=[level distance=1.2cm, sibling distance=1.2cm]
       \tikzstyle{childnode}=[rounded corners, black]
       \tikzstyle{marknode}=[fill=red!30, rounded corners]
       \tikzstyle{edge from parent}=[draw,thick]
       \node {0} 
        child[red, <-] {node[childnode] {1} 
         child {node[marknode] {\Cb{2}}
          child {node[childnode] {3}
           edge from parent node[left] {true}
          }
          edge from parent node[left] {false}
         }
         child {node[marknode] {\Cb{4}}
          child {node[childnode] {5}
           edge from parent node[left] {true}
          }
          child {node[marknode] {\Cb{6}} 
           child {node[childnode] {7}
            edge from parent node[left] {true}
           }
           edge from parent node[right] {false}
          }
          edge from parent node[right] {false}
         }
         edge from parent node[left] {true}
        };
      \end{tikzpicture}
      }\only<19->{
      \begin{tikzpicture}
       \tikzstyle{level 1}=[level distance=1.2cm, sibling distance=1.2cm]
       \tikzstyle{childnode}=[rounded corners, black]
       \tikzstyle{marknode}=[fill=red!30, rounded corners]
       \tikzstyle{edge from parent}=[draw,thick]
       \node[marknode] {\Cb{0}} 
        child[red, <-] {node[childnode] {1} 
         child {node[marknode] {\Cb{2}}
          child {node[childnode] {3}
           edge from parent node[left] {true}
          }
          edge from parent node[left] {false}
         }
         child {node[marknode] {\Cb{4}}
          child {node[childnode] {5}
           edge from parent node[left] {true}
          }
          child {node[marknode] {\Cb{6}} 
           child {node[childnode] {7}
            edge from parent node[left] {true}
           }
           edge from parent node[right] {false}
          }
          edge from parent node[right] {false}
         }
         edge from parent node[left] {true}
        };
      \end{tikzpicture}
      }
     \end{figure}
     \only<20>{Couverture minimale = \{0,2,4,6\}}
    \end{center}
   \end{block}
  \end{frame}
 

 \section{Algorithmes d'approximations}
 %==============================================================================
 \frame{\tableofcontents[current]}
 %% approx.tex %%
  \subsection{Bref rappel des algorithmes}

  \subsection{Comparaison} 
  
 \section{Instances critiques}
 %==============================================================================
 \frame{\tableofcontents[current]}
 %% inst.tex %%
\begin{frame}
 \frametitle{Graphe critique pour les algorithmes d'approximation}
 
 \begin{block}{Données}
  \begin{center}
   \begin{figure}[!ht]
    \begin{tikzpicture}[font=\small]
     \tikzstyle{level 1}=[level distance=0.8cm, sibling distance=0.8cm]
     \tikzstyle{childnode}=[rounded corners]
     \tikzstyle{marknode}=[fill=red!30, rounded corners]
     \tikzstyle{edge from parent}=[draw,thick]
     \node[childnode] {0} 
      child[grow=south east] {node[childnode] {1} 
       child[grow=south west] {node[childnode] {2}
        child[grow=south east] {node[childnode] {3}
         child[grow=south west] {node[childnode] {4}
          child[grow=south east] {node[childnode] {5}
           child[grow=south west] {node[childnode] {6} 
           }
          }
         }
        }
       }
      };
    \end{tikzpicture}

    \caption{Graphe \emph{serpent}}
   \end{figure}
  \end{center}
 \end{block}
\end{frame}

\begin{frame}
 \frametitle{Solution pour le graphe \emph{serpent}}

 \begin{center}
  \begin{block}{Comparaison des solutions sur le graphe \emph{serpent}}
   \begin{figure}[!ht]
    \setlength{\tabcolsep}{1cm}
    \begin{tabular}{cc}
     \begin{tikzpicture}[font=\small]
      \tikzstyle{level 1}=[level distance=0.8cm, sibling distance=0.8cm]
      \tikzstyle{childnode}=[rounded corners]
      \tikzstyle{marknode}=[fill=red!30, rounded corners]
      \tikzstyle{edge from parent}=[draw,thick]
      \node[marknode] {0} 
       child[grow=south east] {node[marknode] {1} 
        child[grow=south west] {node[marknode] {2}
         child[grow=south east] {node[marknode] {3}
          child[grow=south west] {node[marknode] {4}
           child[grow=south east] {node[marknode] {5}
            child[grow=south west] {node[childnode] {6} 
           }
          }
         }
        }
       }
      };
     \end{tikzpicture}
     &
     \begin{tikzpicture}[font=\small]
      \tikzstyle{level 1}=[level distance=0.8cm, sibling distance=0.8cm]
      \tikzstyle{childnode}=[rounded corners]
      \tikzstyle{marknode}=[fill=red!30, rounded corners]
      \tikzstyle{edge from parent}=[draw,thick]
      \node[marknode] {0} 
       child[grow=south east, red] {node[marknode] {\Cb{1}} 
        child[grow=south west, black] {node[marknode] {2}
         child[grow=south east, red] {node[marknode] {\Cb{3}}
          child[grow=south west, black] {node[marknode] {4}
           child[grow=south east, red] {node[marknode] {\Cb{5}}
            child[grow=south west, black] {node[childnode] {6} 
            }
           }
          }
         }
        }
       };
     \end{tikzpicture}\\
     $C_{projet} = \{0,1,2,3,4,5\}$ & $C_{cours} = \{0,1,2,3,4,5\}$\\
    \end{tabular}
   \end{figure}    
  \end{block}
 \end{center}
\end{frame}

\begin{frame}
 \frametitle{Graphe critique pour les algorithmes d'approximation}
 
 \begin{block}{Couverture minimale surle graphe \emph{serpent}}
  \begin{center}
   \begin{figure}[!ht]
    \begin{tikzpicture}[font=\small]
     \tikzstyle{level 1}=[level distance=0.8cm, sibling distance=0.8cm]
     \tikzstyle{childnode}=[rounded corners]
     \tikzstyle{marknode}=[fill=red!30, rounded corners]
     \tikzstyle{edge from parent}=[draw,thick]
     \node[childnode] {0} 
      child[grow=south east] {node[marknode] {1} 
       child[grow=south west] {node[childnode] {2}
        child[grow=south east] {node[marknode] {3}
         child[grow=south west] {node[childnode] {4}
          child[grow=south east] {node[marknode] {5}
           child[grow=south west] {node[childnode] {6} 
           }
          }
         }
        }
       }
      };
    \end{tikzpicture}
   \end{figure}
   Couverture minimale = $\{1,3,5\}$
  \end{center}
 \end{block}
\end{frame}
 

 \section{Bilan}
 %==============================================================================
 \frame{\tableofcontents[current]}
 %% bilan.tex %%
\begin{frame}
 \frametitle{Bilan}

 \begin{block}{Améliorations}
  \begin{itemize}
   \item Partie \emph{Réduction} : 
	 \begin{itemize}
	  \item Utiliser des structures de graphes différentes selon les
		réductions.
	  \item Trouver plus de \emph{``cas faciles''}
	 \end{itemize}
   \item<2-> Partie \emph{Approximation} :
	 \begin{itemize}
	  \item Utiliser une structure de graphe plus adaptée à la
		suppression des arêtes pour le couplage maximum.
	  \item Améliorer le \emph{parser} qui est la cause d'exécutions
		lentes.
	 \end{itemize}
  \end{itemize}
 \end{block}

 \begin{block}{Conclusion}<3->
  \begin{itemize}
   \item Nous avons remarqué l'intêret d'utiliser des algorithmes
	 d'approximation lorsque le calcul d'une solution exacte est
	 trop coûteux
   \item Un des rares projets où la théorie est au service de la
	 pratique
  \end{itemize}
 \end{block}
\end{frame}


%  \section*{Références}
%  \begin{frame}[allowframebreaks]
%   \frametitle{Quelques références}
%   \nocite{*}
%   \bibliographystyle{alpha}
%   \bibliography{biblio}
%  \end{frame}

%%%%%%               %%%%%%
%%%%%% /ZONE DE CODE %%%%%%
%%%%%%               %%%%%%

\end{document}



%%%%%%           %%%%%%
%%%%%% /DOCUMENT %%%%%%
%%%%%%           %%%%%%




%%%%% Texte italique %%%%%
%  \textit{} 

%%%%% Liste %%%%
%%% Itemize %%%
%  \begin{itemize}
%   \item item1\\
%   \item item2
%  \end{itemize}

%%% Enumerate %%%
%  \begin{enumerate}
%   \item item1
%   \item item2
%  \end{enumerate}

%%%%% Tabulations %%%%%
%   \begin{tabbing}
%    XX\=XX\=\kill
%    \>(OrdresEnonce.v, ligne 244)\\
%    \>\>test2
%   \end{tabbing}
 
%%%%% Note de pied de page %%%%%
%  \footnote{test}

%%%%% Référence %%%%%
%   \label{ref}
%% Plus loin :
%   \ref{ref}
%   \pageref{ref}

%%%%% Code %%%%%
% \begin{lstlisting}
%      List<Integer> lexBFS2 = new ArrayList<Integer>();
%      lexBFS2.add(3);
%      lexBFS2.add(2);
%      lexBFS2.add(4);
%      lexBFS2.add(1);
%      lexBFS2.add(0);    
%      assertNotNull(lexBFS2);	
%      assertEquals(lexBFS2,Graphs.lexBFS(nogYComp));
%   \end{lstlisting}

%%%%% Inclure une figure %%%%%
%  \begin{figure}[!ht]
%   \begin{center}
%	\includegraphics[width=7cm]{figs/cours1/fig2.eps}
%	\caption{\emph{MT2} : Calcul non déterministe}
%   \end{center}
%  \end{figure}

%%%%% Dessiner une figure %%%%% 
%   \begin{center}
%    \begin{tikz_mrfou}

%     %% Nodes %%
%     \node[state, initial] (1) {$1$};
%     \node[state, right of=1] (2) {$2$};
%     \node[state, right of=2] (3) {$3$};

%     %% Edges %%
%     \path[->] 
%     (1) 
%     edge [loop above] node {b} ()
%     edge node {a} (2)

%     (2)
%     edge [loop above] node {b} ()
%     edge [bend left] node {a} (3)

%     (3)
%     edge [bend left] node {b} (2);

%    \end{tikz_mrfou}
%   \end{center}
