%%%%% Remove ? to make this file compilable %%%%%
% TO COMPILE %

%%%%%%%%%%                             %%%%%%%%%%
%%%%%%%%%% AUTO INSERTION D'UNE ENTETE %%%%%%%%%%
%%%%%%%%%%                             %%%%%%%%%%
%%%%%%%%%%     POUR UN FICHIER TEX     %%%%%%%%%%
%%%%%%%%%%                             %%%%%%%%%%

%%%%%%          %%%%%%
%%%%%% PACKAGES %%%%%%
%%%%%%          %%%%%%

\documentclass[8pt]{beamer}
\usepackage[utf8]{inputenc}
\usepackage[french]{babel}
\usepackage[babel=true,kerning=true]{microtype} % probleme avec : tikz
\usepackage{graphicx}
\usepackage{listings}
\usepackage{color} 
\usepackage{enumerate}
\usepackage{amsfonts}
\usepackage{amssymb}
\usepackage{amsmath}
\usepackage{wasysym}
\usepackage{chemarrow}
\usepackage{tikz}
\usetikzlibrary{positioning}
\usetikzlibrary{automata}
\usetikzlibrary{trees}
\usetikzlibrary{arrows}
\usepackage{beamerthemesplit}
\usetheme{JuanLesPins}



%%%%%%              %%%%%%
%%%%%% MISE EN PAGE %%%%%%
%%%%%%              %%%%%%


%%%%% Environnements %%%%%

\newenvironment{tikz_mrfou}{
\begin{tikzpicture}[->,>=stealth,shorten >=1pt,auto,node
 distance=1.5cm,on grid, semithick, minimum size=5mm,bend
 angle=10,font=\small, initial text=, label distance=-1mm]
}{\end{tikzpicture}}

\tikzstyle{evennode}=[circle,draw=blue!50,fill=blue!20,thick, inner
sep=0pt,minimum size=6mm]

\tikzstyle{oddnode}=[rectangle,draw=black!50,fill=black!20,thick, inner
sep=0pt,minimum size=5mm]

\tikzstyle{pointnode}=[circle,draw=black,fill=black, thick, inner
sep=0pt,minimum size=1mm] 

%%%%%%          %%%%%%
%%%%%% DOCUMENT %%%%%%
%%%%%%          %%%%%%


\begin{document}

%%%%%%              %%%%%%
%%%%%% ZONE DE CODE %%%%%%
%%%%%%              %%%%%%

%% Style bloc %%
\setbeamertemplate{blocks}[rounded][shadow=true]
\setbeamercovered{dynamic}

%% Page de garde %%
\author[Ludovic Brochard, Fabien Kuntz, Benoît Védrenne]{Ludovic
Brochard, Fabien Kuntz, Benoît Védrenne\\ 
~\\
\small Professeurs : Cyril Gavoille, Anca Muscholl, Marc Zeitoun,
Alexander Zvonkin}
\institute{\footnotesize Module AMR\\~\\Master S\&T
Informatique\\~\\Université Bordeaux I\\} 
\title{Soutenances d'\textit{Algorithmes du Monde Réel}} 
\date{22 janvier 2009 - 11h40}

\setcounter{page}{1}

%% Debut %%
\frame{\titlepage}
\frame{\tableofcontents}

 \section{Introduction}
 %==============================================================================
 \frame{\tableofcontents[current]}
  %==============================================================================
 \begin{frame}
  \frametitle{Le projet}
  
  \begin{block}{Principe}
   \begin{itemize}
    \item Deux robots : Maître et Esclave
	  \uncover<2->{
    \item Robot Maître : capteurs et moteurs}
	  \uncover<3->{
    \item Robot Esclave : seulement moteurs}
	  \uncover<4->{
    \item Maître communique les ordres à l'Esclave par Bluetooth}
	  \uncover<5->{
    \item Esclave : pas de décision}
   \end{itemize}
  \end{block}

  \uncover<6->{
  \begin{block}{But}
   \begin{itemize}
    \item Accomplir une mission non triviale mettant en jeu les deux robots
	  \uncover<7->{
    \item Vérifier cette mission en modélisant en Altarica}
    \uncover<8->{
    \item Etudier la possibilité d'une generation de code automatique}
   \end{itemize}
  \end{block}
  }

 \end{frame}


 \section{Réductions}
 %==============================================================================
 \frame{\tableofcontents[current]}
 %% reduc.tex %%

 \section{Couverture par sommets}
 %==============================================================================
 \frame{\tableofcontents[current]}
 %% vc.tex %% 

 \section{Algorithmes d'approximations}
 %==============================================================================
 \frame{\tableofcontents[current]}
 %% approx.tex %%
  \subsection{Bref rappel des algorithmes}
  \begin{frame}
   \frametitle{Rappel}

   \begin{block}{Algorithme du projet (sommets non feuilles)}
    \begin{itemize}
     \item Calcul d'un arbre couvrant par parcours en profondeur
     \item Couverture = ensemble des sommets non feuilles (par parcours
	   en profondeur)
    \end{itemize}
   \end{block}

   \begin{block}{Algorithme du cours (couplage maximal)}
    \begin{itemize}
     \item Couverture = ensemble des sommets d'un couplage maximal
    \end{itemize}
   \end{block}

   \begin{block}{Complexités}
    \begin{itemize}
     \item Algorithme du projet : linéaire car consiste en deux parcours
	   en profondeur l'un après l'autre.
     \item Avec une structure de graphe adapté : linéaire.
    \end{itemize}
   \end{block}
  \end{frame}

  \subsection{Comparaison}
  \begin{frame}
   \frametitle{Comparaison des algorithmes d'approximation (1/3)}
   
   \begin{center}
    \begin{block}{Données : Graphe de comparaison \emph{Pise}}
     \begin{figure}[!ht]
      \begin{tikzpicture}[font=\small]
       \tikzstyle{level 1}=[level distance=0.8cm, sibling distance=0.8cm]
       \tikzstyle{childnode}=[rounded corners]
       \tikzstyle{marknode}=[fill=red!30, rounded corners]
       \tikzstyle{edge from parent}=[draw,thick]
       \node {0} 
        child {node[childnode] {1} 
         child {node[childnode] {2}
          child {node[childnode] {3}
          }
         }
         child {node[childnode] {4}
          child {node[childnode] {5}
          }
          child {node[childnode] {6} 
           child {node[childnode] {7}
           }
           child {node[childnode] {8}
            child {node[childnode] {9}
            }
            child {node[childnode] {10}
             child {node[childnode] {11}
             }
             child {node[childnode] {12}
              child {node[childnode] {13}
              }
             }
            }
           }
          }
         }
        };
      \end{tikzpicture}
     \end{figure}
    \end{block}
   \end{center}
  \end{frame}

  \begin{frame}
   \frametitle{Comparaison des algorithmes d'approximation (2/3)}

   \setlength{\tabcolsep}{1cm}
   \begin{center}
    \begin{block}{Comparaison des solutions sur le graphe \emph{Pise}}
     \begin{tabular}{cc}
      \begin{tikzpicture}[font=\small]
       \tikzstyle{level 1}=[level distance=0.8cm, sibling distance=0.8cm]
       \tikzstyle{childnode}=[rounded corners]
       \tikzstyle{marknode}=[fill=red!30, rounded corners]
       \tikzstyle{edge from parent}=[draw,thick]
       \node[childnode, fill=red!30] {0} 
        child {node[childnode,fill=red!30] {1} 
         child {node[childnode,fill=red!30] {2}
          child {node[childnode] {3}
          }
         }
         child {node[childnode,fill=red!30] {4}
          child {node[childnode] {5}
          }
          child {node[childnode,fill=red!30] {6} 
           child {node[childnode] {7}
           }
           child {node[childnode,fill=red!30] {8}
            child {node[childnode] {9}
            }
            child {node[childnode,fill=red!30] {10}
             child {node[childnode] {11}
             }
             child {node[childnode,fill=red!30] {12}
              child {node[childnode] {13}
              }
             }
            }
           }
          }
         }
        };
      \end{tikzpicture}
      &
      \begin{tikzpicture}[font=\small]
       \tikzstyle{level 1}=[level distance=0.8cm, sibling distance=0.8cm]
       \tikzstyle{childnode}=[rounded corners, fill=red!30]
       \tikzstyle{marknode}=[fill=red!30, rounded corners]
       \tikzstyle{edge from parent}=[draw,thick]
       \node[childnode] {0} 
        child[red] {node[childnode] {\Cb{1}} 
         child[black] {node[childnode] {\Cb{2}}
          child[red] {node[childnode] {\Cb{3}}
          }
         }
         child[black] {node[childnode] {\Cb{4}}
          child[red] {node[childnode] {\Cb{5}}
          }
          child[black] {node[childnode] {\Cb{6}}
           child[red] {node[childnode] {\Cb{7}}
           }
           child[black] {node[childnode] {\Cb{8}}
            child[red] {node[childnode] {\Cb{9}}
            }
            child[black] {node[childnode] {\Cb{10}}
             child[red] {node[childnode] {\Cb{11}}
             }
             child[black] {node[childnode] {\Cb{12}}
              child[red] {node[childnode] {\Cb{13}}
              }
             }
            }
           }
          }
         }
        };
      \end{tikzpicture}\\
      $C_{projet} = \{0,1,2,4,6,8,10,12\}$ & $C_{cours} = V$\\
     \end{tabular}
    \end{block}
   \end{center}
  \end{frame}

  \begin{frame}
   \frametitle{Comparaison des algorithmes d'approximation (3/3)}

   \begin{block}{Tableau de comparaison}
    \begin{figure}[!ht]
     \begin{center}
      \footnotesize
      \begin{tabular}{|c|c|c||c|c||c|c|}
       \cline{2-7}
       \multicolumn{1}{c|}{} & \multicolumn{2}{|c||}{jeuPise.gin}
       &\multicolumn{2}{|c||}{testTree100.gin} &
       \multicolumn{2}{|c|}{testTree1000.gin}\\ 
       \cline{2-7}
       \multicolumn{1}{c|}{} & algoP & algoC & algoP & algoC & algoP &
       algoC\\
       \hline
       Temps d'execution(en $\mu$s) & 58.5808 & 1.12014 & 400.344 &
       5.82175 & 3911.25 & 56.8837\\
       \hline
       Taille de la couverture & 8 & 14 & 38 & 58 & 365 & 576\\
       \hline
       \hline
       Taille couverture optimale & \multicolumn{2}{|c||}{7} &
		   \multicolumn{2}{|c||}{31} &
       \multicolumn{2}{|c|}{325}\\ 
       \hline
      \end{tabular}
      \caption{Tableau comparatif des exécutions des algorithmes sur
      plusieurs exemples.\label{tableau}} 
     \end{center}
    \end{figure}  
   \end{block}
  \end{frame}
 
  
 \section{Instances critiques}
 %==============================================================================
 \frame{\tableofcontents[current]}
 %% inst.tex %%
\begin{frame}
 \frametitle{Graphe critique pour les algorithmes d'approximation}
 
 \begin{block}{Données}
  \begin{center}
   \begin{figure}[!ht]
    \begin{tikzpicture}[font=\small]
     \tikzstyle{level 1}=[level distance=0.8cm, sibling distance=0.8cm]
     \tikzstyle{childnode}=[rounded corners]
     \tikzstyle{marknode}=[fill=red!30, rounded corners]
     \tikzstyle{edge from parent}=[draw,thick]
     \node[childnode] {0} 
      child[grow=south east] {node[childnode] {1} 
       child[grow=south west] {node[childnode] {2}
        child[grow=south east] {node[childnode] {3}
         child[grow=south west] {node[childnode] {4}
          child[grow=south east] {node[childnode] {5}
           child[grow=south west] {node[childnode] {6} 
           }
          }
         }
        }
       }
      };
    \end{tikzpicture}

    \caption{Graphe \emph{serpent}}
   \end{figure}
  \end{center}
 \end{block}
\end{frame}

\begin{frame}
 \frametitle{Solution pour le graphe \emph{serpent}}

 \begin{center}
  \begin{block}{Comparaison des solutions sur le graphe \emph{serpent}}
   \begin{figure}[!ht]
    \setlength{\tabcolsep}{1cm}
    \begin{tabular}{cc}
     \begin{tikzpicture}[font=\small]
      \tikzstyle{level 1}=[level distance=0.8cm, sibling distance=0.8cm]
      \tikzstyle{childnode}=[rounded corners]
      \tikzstyle{marknode}=[fill=red!30, rounded corners]
      \tikzstyle{edge from parent}=[draw,thick]
      \node[marknode] {0} 
       child[grow=south east] {node[marknode] {1} 
        child[grow=south west] {node[marknode] {2}
         child[grow=south east] {node[marknode] {3}
          child[grow=south west] {node[marknode] {4}
           child[grow=south east] {node[marknode] {5}
            child[grow=south west] {node[childnode] {6} 
           }
          }
         }
        }
       }
      };
     \end{tikzpicture}
     &
     \begin{tikzpicture}[font=\small]
      \tikzstyle{level 1}=[level distance=0.8cm, sibling distance=0.8cm]
      \tikzstyle{childnode}=[rounded corners]
      \tikzstyle{marknode}=[fill=red!30, rounded corners]
      \tikzstyle{edge from parent}=[draw,thick]
      \node[marknode] {0} 
       child[grow=south east, red] {node[marknode] {\Cb{1}} 
        child[grow=south west, black] {node[marknode] {2}
         child[grow=south east, red] {node[marknode] {\Cb{3}}
          child[grow=south west, black] {node[marknode] {4}
           child[grow=south east, red] {node[marknode] {\Cb{5}}
            child[grow=south west, black] {node[childnode] {6} 
            }
           }
          }
         }
        }
       };
     \end{tikzpicture}\\
     $C_{projet} = \{0,1,2,3,4,5\}$ & $C_{cours} = \{0,1,2,3,4,5\}$\\
    \end{tabular}
   \end{figure}    
  \end{block}
 \end{center}
\end{frame}

\begin{frame}
 \frametitle{Graphe critique pour les algorithmes d'approximation}
 
 \begin{block}{Couverture minimale surle graphe \emph{serpent}}
  \begin{center}
   \begin{figure}[!ht]
    \begin{tikzpicture}[font=\small]
     \tikzstyle{level 1}=[level distance=0.8cm, sibling distance=0.8cm]
     \tikzstyle{childnode}=[rounded corners]
     \tikzstyle{marknode}=[fill=red!30, rounded corners]
     \tikzstyle{edge from parent}=[draw,thick]
     \node[childnode] {0} 
      child[grow=south east] {node[marknode] {1} 
       child[grow=south west] {node[childnode] {2}
        child[grow=south east] {node[marknode] {3}
         child[grow=south west] {node[childnode] {4}
          child[grow=south east] {node[marknode] {5}
           child[grow=south west] {node[childnode] {6} 
           }
          }
         }
        }
       }
      };
    \end{tikzpicture}
   \end{figure}
   Couverture minimale = $\{1,3,5\}$
  \end{center}
 \end{block}
\end{frame}
 

 \section{Bilan}
 %==============================================================================
 \frame{\tableofcontents[current]}
 %% bilan.tex %%
\begin{frame}
 \frametitle{Bilan}

 \begin{block}{Améliorations}
  \begin{itemize}
   \item Partie \emph{Réduction} : 
	 \begin{itemize}
	  \item Utiliser des structures de graphes différentes selon les
		réductions.
	  \item Trouver plus de \emph{``cas faciles''}
	 \end{itemize}
   \item<2-> Partie \emph{Approximation} :
	 \begin{itemize}
	  \item Utiliser une structure de graphe plus adaptée à la
		suppression des arêtes pour le couplage maximum.
	  \item Améliorer le \emph{parser} qui est la cause d'exécutions
		lentes.
	 \end{itemize}
  \end{itemize}
 \end{block}

 \begin{block}{Conclusion}<3->
  \begin{itemize}
   \item Nous avons remarqué l'intêret d'utiliser des algorithmes
	 d'approximation lorsque le calcul d'une solution exacte est
	 trop coûteux
   \item Un des rares projets où la théorie est au service de la
	 pratique
  \end{itemize}
 \end{block}
\end{frame}


%  \section*{Références}
%  \begin{frame}[allowframebreaks]
%   \frametitle{Quelques références}
%   \nocite{*}
%   \bibliographystyle{alpha}
%   \bibliography{biblio}
%  \end{frame}

%%%%%%               %%%%%%
%%%%%% /ZONE DE CODE %%%%%%
%%%%%%               %%%%%%

\end{document}



%%%%%%           %%%%%%
%%%%%% /DOCUMENT %%%%%%
%%%%%%           %%%%%%




%%%%% Texte italique %%%%%
%  \textit{} 

%%%%% Liste %%%%
%%% Itemize %%%
%  \begin{itemize}
%   \item item1\\
%   \item item2
%  \end{itemize}

%%% Enumerate %%%
%  \begin{enumerate}
%   \item item1
%   \item item2
%  \end{enumerate}

%%%%% Tabulations %%%%%
%   \begin{tabbing}
%    XX\=XX\=\kill
%    \>(OrdresEnonce.v, ligne 244)\\
%    \>\>test2
%   \end{tabbing}
 
%%%%% Note de pied de page %%%%%
%  \footnote{test}

%%%%% Référence %%%%%
%   \label{ref}
%% Plus loin :
%   \ref{ref}
%   \pageref{ref}

%%%%% Code %%%%%
% \begin{lstlisting}
%      List<Integer> lexBFS2 = new ArrayList<Integer>();
%      lexBFS2.add(3);
%      lexBFS2.add(2);
%      lexBFS2.add(4);
%      lexBFS2.add(1);
%      lexBFS2.add(0);    
%      assertNotNull(lexBFS2);	
%      assertEquals(lexBFS2,Graphs.lexBFS(nogYComp));
%   \end{lstlisting}

%%%%% Inclure une figure %%%%%
%  \begin{figure}[!ht]
%   \begin{center}
%	\includegraphics[width=7cm]{figs/cours1/fig2.eps}
%	\caption{\emph{MT2} : Calcul non déterministe}
%   \end{center}
%  \end{figure}

%%%%% Dessiner une figure %%%%% 
%   \begin{center}
%    \begin{tikz_mrfou}

%     %% Nodes %%
%     \node[state, initial] (1) {$1$};
%     \node[state, right of=1] (2) {$2$};
%     \node[state, right of=2] (3) {$3$};

%     %% Edges %%
%     \path[->] 
%     (1) 
%     edge [loop above] node {b} ()
%     edge node {a} (2)

%     (2)
%     edge [loop above] node {b} ()
%     edge [bend left] node {a} (3)

%     (3)
%     edge [bend left] node {b} (2);

%    \end{tikz_mrfou}
%   \end{center}
