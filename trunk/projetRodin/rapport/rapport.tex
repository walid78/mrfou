\documentclass{article}
\usepackage[utf8]{inputenc}
\usepackage[francais]{babel}
\usepackage{amsmath}
\usepackage{graphicx}
\usepackage{listings}
\usepackage{color} 
\usepackage{enumerate}
\usepackage[colorlinks=true, urlcolor=blue]{hyperref}
\usepackage[top=3cm, bottom=2.5cm, left=3cm, right=2.5cm] {geometry}
\usepackage {latex/bsymb,latex/b2latex}
\usepackage{fancyhdr}

%%%%% Mise en page %%%%%
%\addtolength{\textwidth}{3cm}
%\addtolength{\oddsidemargin}{-3cm}
%\textheight=22cm % longueur utile de la page
%\topmargin=-2cm % marge haute
%\headsep=20pt % séparation 20 points entre entête et texte
%\footskip=20pt % idem séparation pied de page

%%%%% Entetes et Enpieds de pages %%%%%
\pagestyle{fancy}
\lhead{\footnotesize{\textit{Hannier Julien, Kuntz Fabien}}}
\rhead{\footnotesize{\textit{Projet de Conceptions Formelles}}}

%%%%% Numérotation sections %%%%
\setcounter{secnumdepth}{3}
\setcounter{tocdepth}{3}

%%%%% Titre %%%%%
\newlength{\larg}
\setlength{\larg}{14.5cm}

\title{
{\rule{\larg}{1mm}}\vspace{7mm}
\begin{flushright}
 \Huge{\bf Implémenter un service de Hotline} \\
 \Huge{\bf en Event-B avec Rodin} \\
 ~\\
 \huge{Projet de Conceptions Formelles}
\end{flushright}
\vspace{2mm}
{\rule{\larg}{1mm}}
\vspace{2mm} \\
\begin{flushright}
 \large{\bf Rapport} \\
 ~\\
 \large{Chargé de TD : Alain Griffault}\\
 ~\\
 \today
\end{flushright}
\vspace{9.5cm}
}
\author{\begin{tabular}{p{13.7cm}}
Hannier Julien, Kuntz Fabien.
\end{tabular}\\
\hline }
\date{}
%%%Zone de Code%%%
\definecolor{gray}{gray}{0.86}  

\lstset{numbers=left, tabsize=2, frame=single, breaklines=true,
basicstyle=\ttfamily,numberstyle=\tiny\ttfamily, framexleftmargin=13mm,
backgroundcolor=\color{gray}, xleftmargin=14mm} 

\newsavebox{\fmbox}
\newenvironment{fmpage}[1]{\begin{lrbox}{\fmbox}\begin{minipage}{#1}}{\end{minipage}\end{lrbox}\fbox{\usebox{\fmbox}}}


\begin{document}

%%%%% Titre %%%%%
\thispagestyle{empty}
\maketitle
\newpage

%%%%% Page blanche %%%%%
\thispagestyle{empty}
~\newpage

%%%%% Table des matières %%%%%
\thispagestyle{empty}
\tableofcontents
\newpage

\setcounter{page}{1}

 \section{Présentation du Projet}
 
  \subsection{Cahier des charges}
  Un vendeur de logiciels propose à ses clients d'utiliser un service de
  Hotline lorsqu'ils rencontrent un problème avec ses
  produits.\newline
  
  La procédure est simple : En cas de panne, ils doivent appeler un
  numéro téléphonique. Un(e) opérateur(rice) hautement qualifié(e) leur
  indiquera alors la démarche à suivre. \newline
  
  \subsection{Contraintes}
  \begin{itemize}
   \item Le nombre d'opérateurs(rices) est fini.
   \item Le service doit être toujours opérationnel.
   \item Le service doit être le plus équitable possible.
   \item L'employeur doit respecter le code du travail qui prévoit des
		 pauses pour les opérateurs(trices).
  \end{itemize}

  \subsection{Dysfonctionnements}
  Une communication peut pour des raisons diverses être
  interrompue. Dans ce cas, le client (sans doute alors très mécontent)
  doit être très rapidement servi, mais pas nécessairement par le(a)
  même opérateur(rice).

  \subsection{Objectif}
  L'objectif est de construire par raffinements successifs, un
  environnement certifié avec l'outil \emph{Rodin} modélisant ce
  service.\newline

 \section{Présentation des contextes}
 Les fichiers suivants définissent les ensembles et constantes
 nécessaires à la modélisation du système. 
 
  \subsection{Rôle des contextes}
  Le contexte \emph{Clients} définit notre ensemble de clients.\newline
  
  \indent Le contexte \emph{Central} définit notre ensemble d'opérateurs
  ainsi que leur état.\newline
  Les différents états sont \emph{pause} quand l'opérateur est en pause,
  \emph{reponse} quand l'opérateur est au téléphone et \emph{lire} quand
  l'opérateur est en attente d'un appel.\newline
  Le nombre d'opérateurs est fini. Il est désigné par la constante
  \emph{m}.

  \indent Le contexte \emph{droitDuTravail} définit le nombre maximum de
  pauses que peut prendre un opérateur. 

  \subsection{Code des contextes}

  \thispagestyle{empty}
\begin{description}
\BTitle{Clients}{11 Apr 2008}{04:17:03 PM}
\CONTEXT{Clients}
\SETS
	\begin{description}
		\Item{ CLIENTS }
	\end{description}
\AXIOMS
	\begin{description}
		\nItem{ axm1 }{ card(CLIENTS)>  0 }
	\end{description}
\END
\end{description}

  \thispagestyle{empty}
\begin{description}
\BTitle{Central}{11 Apr 2008}{04:16:59 PM}
\CONTEXT{Central}
\SETS
	\begin{description}
		\Item{ Operateurs }
		\Item{ Etat }
	\end{description}
\CONSTANTS
	\begin{description}
		\Item{ pause }
		\Item{ reponse }
		\Item{ libre }
		\Item{ m }
		\bcmt{ Le nombre d'opérateurs }
	\end{description}
\AXIOMS
	\begin{description}
		\nItem{ axm1 }{ Etat = \{ pause, reponse, libre\}  }
		\nItem{ axm2 }{ pause \neq  reponse }
		\nItem{ axm3 }{ pause \neq  libre }
		\nItem{ axm4 }{ reponse \neq  libre }
		\nItem{ axm5 }{ m \in  \nat_1 }
		\nItem{ axm6 }{ finite(Operateurs) }
		\nItem{ axm7 }{ card(Operateurs) = m }
	\end{description}
\THEOREMS
	\begin{description}
		\nItem{ thm1 }{ finite(Etat) }
	\end{description}
\END
\end{description}

  \thispagestyle{empty}
\begin{description}
\BTitle{droitDuTravail}{11 Apr 2008}{04:17:05 PM}
\CONTEXT{droitDuTravail}
\CONSTANTS
	\begin{description}
		\Item{ pauseMax }
	\end{description}
\AXIOMS
	\begin{description}
		\nItem{ axm1 }{ pauseMax\in \nat_1  }
	\end{description}
\END
\end{description}


 \section{Présentation des machines}
 
  \subsection{Machine H0}
  
   \subsubsection{Rôle de la machine H0}
   La première machine consiste à prendre le système du côté du
   client. L'ensemble \emph{CLIENTS} est alors découpé en deux
   sous-ensembles : 

   \begin{itemize} 
	\item \emph{OK} quand le client est satisfait.
	\item \emph{Lose} quand son logiciel ne marche pas. 
   \end{itemize}

   \indent Ainsi, lors d'une panne de logiciel, le client passe dans le
   sous-ensemble \emph{Lose} et inversement quand sa panne est réparée.

   \thispagestyle{empty}
\begin{description}
\BTitle{H0}{11 Apr 2008}{04:17:08 PM}
\MACHINE{H0}
\SEES{Clients}
\VARIABLES
	\begin{description}
		\Item{ Ok }
		\Item{ Lose }
	\end{description}
\INVARIANTS
	\begin{description}
		\nItem{ inv1 }{ Ok\subseteq CLIENTS }
		\nItem{ inv2 }{ Lose\subseteq CLIENTS }
		\nItem{ inv3 }{ Ok\bunion Lose=CLIENTS }
		\nItem{ inv4 }{ Ok\binter Lose=\emptyset  }
	\end{description}
\EVENTS
	\INITIALISATION
		\begin{description}
		\Begin
			\begin{description}
			\nItem{ act1 }{ Ok:= CLIENTS }
			\nItem{ act2 }{ Lose:= \emptyset  }
			\end{description}
		\End
		\end{description}
	\EVT {Panne}
		\begin{description}
		\Any
			\begin{description}
			\Item{ c }
			\end{description}
		\Where
			\begin{description}
			\nItem{ grd1 }{ c\in Ok }
			\end{description}
		\Then
			\begin{description}
			\nItem{ act1 }{ Ok,Lose:= Ok\setminus \{ c\} ,Lose\bunion \{ c\}  }
			\end{description}
		\End
		\end{description}
	\EVT {Fix}
		\begin{description}
		\Any
			\begin{description}
			\Item{ c }
			\end{description}
		\Where
			\begin{description}
			\nItem{ grd1 }{ c\in Lose }
			\end{description}
		\Then
			\begin{description}
			\nItem{ act1 }{ Lose,Ok:= Lose\setminus \{ c\} ,Ok\bunion \{ c\}  }
			\end{description}
		\End
		\end{description}
\END
\end{description}


  \subsection{Machine H1}
  
   \subsubsection{Rôle de la machine H1}
   Ce premier raffinement précise un peu plus le système.\newline

   \indent Le sous-ensemble \emph{Lose} est à son tour découpé en deux
   sous-ensembles \emph{Frustrated} et \emph{Wait}. Le client dont le
   logiciel tombe en panne passe donc de \emph{Ok} à
   \emph{Frustrated}. Ensuite, il appelle la \emph{hotline} et va alors
   dans \emph{Wait}. Ici, deux événements sont possibles :
   \begin{itemize} 
	\item Il peut survenir une coupure durant la communication. À ce
		  moment là, le client repasse dans \emph{Frustrated}, auquel
		  cas il devra rappeler.
	\item Ou alors la panne du client est réparée et il va alors dans le
		  sous-ensemble \emph{Ok}.
   \end{itemize}

   \thispagestyle{empty}
\begin{description}
\BTitle{H1}{11 Apr 2008}{04:17:12 PM}
\MACHINE{H1}
\REFINES{H0}
\SEES{Clients}
\VARIABLES
	\begin{description}
		\Item{ Ok }
		\Item{ Wait }
		\Item{ Frustrated }
	\end{description}
\INVARIANTS
	\begin{description}
		\nItem{ inv1 }{ Wait \subseteq  Lose }
		\nItem{ inv2 }{ Frustrated \subseteq  Lose }
		\nItem{ inv3 }{ Wait \binter  Frustrated = \emptyset  }
		\nItem{ inv4 }{ Lose = Wait \bunion  Frustrated }
		\nItem{ inv5 }{ finite(Wait) }
	\end{description}
\EVENTS
	\INITIALISATION
		\begin{description}
		\Begin
			\begin{description}
			\nItem{ act1 }{ Ok:= CLIENTS }
			\nItem{ act2 }{ Wait,Frustrated:= \emptyset ,\emptyset  }
			\end{description}
		\End
		\end{description}
	\EVT {Panne}
	\REF{Panne}
		\begin{description}
		\Any
			\begin{description}
			\Item{ c }
			\end{description}
		\Where
			\begin{description}
			\nItem{ grd1 }{ c\in Ok }
			\end{description}
		\Then
			\begin{description}
			\nItem{ act1 }{ Ok,Frustrated:= Ok\setminus \{ c\} ,Frustrated \bunion  \{ c\}  }
			\end{description}
		\End
		\end{description}
	\EVT {Fix}
	\REF{Fix}
		\begin{description}
		\Any
			\begin{description}
			\Item{ c }
			\end{description}
		\Where
			\begin{description}
			\nItem{ grd1 }{ c\in Wait }
			\end{description}
		\Then
			\begin{description}
			\nItem{ act1 }{ Ok,Wait := Ok\bunion \{ c\} ,Wait\setminus \{ c\}  }
			\end{description}
		\End
		\end{description}
	\EVT {Clac}
		\begin{description}
		\Any
			\begin{description}
			\Item{ c }
			\end{description}
		\Where
			\begin{description}
			\nItem{ grd1 }{ c\in Wait }
			\end{description}
		\Then
			\begin{description}
			\nItem{ act1 }{ Wait,Frustrated := Wait\setminus \{ c\} ,Frustrated\bunion \{ c\}  }
			\end{description}
		\End
		\end{description}
	\EVT {Call}
		\begin{description}
		\Any
			\begin{description}
			\Item{ c }
			\end{description}
		\Where
			\begin{description}
			\nItem{ grd1 }{ c\in Frustrated }
			\end{description}
		\Then
			\begin{description}
			\nItem{ act1 }{ Wait,Frustrated:= Wait\bunion \{ c\} ,Frustrated\setminus \{ c\}  }
			\end{description}
		\End
		\end{description}
\END
\end{description}


  \subsection{Machine H2}
  
   \subsubsection{Rôle de la machine H2}
   Dans cette machine l'interaction entre clients et opérateurs est
   ajoutée. Il faut donc trouver un moyen de \og lier \fg\ un opérateur
   à un client, ainsi qu'un moyen de changer l'état de
   l'opérateur.\newline
   
   \indent Un opérateur a obligatoirement un état. Nous donc avons défini une
   fonction totale de \emph{Operateurs} dans \emph{Etat}.\newline
   
   \indent Au contraire, un opérateur n'est pas toujours lié à un
   client. Nous utilisons donc une fonction partielle de
   \emph{Operateurs} dans \emph{Wait} pour attribuer un client à un
   opérateur. \newline 

   \indent Pour que la modélisation fonctionne, nous avons rajouté un
   évènement \emph{Answer} qui correspond à la prise en charge d'un
   client par un opérateur. C'est à ce moment là qu'un client est lié à
   un opérateur et que l'état de l'opérateur change à
   \emph{reponse}.\newline

   \indent Nous avons aussi logiquement raffiné les événements
   \emph{Clac} et \emph{Fix} pour détacher un client d'un opérateur et
   changer l'état de l'opérateur. \newline 

   \indent Nous avons ensuite ajouté la gestion des pauses des
   opérateurs suivant le droit du travail. Ainsi, nous avons rajouté une
   fonction totale qui attribue un nombre de pauses à un opérateur ainsi
   que deux événements \emph{Break} et \emph{UnBreak}. L'événement
   \emph{UnBreak} change juste l'opérateur d'état. L'évènement
   \emph{Break}, en plus de changer l'opérateur d'état, regarde si
   celui-ci peut prendre une pause tout en suivant le respect de ses 
   collègues. En effet, un opérateur ne peut prendre une pause que s'il
   y a plus d'opérateurs libres que de clients en attente de prise en
   charge.

   \thispagestyle{empty}
\begin{description}
\BTitle{H2}{11 Apr 2008}{04:17:15 PM}
\MACHINE{H2}
\REFINES{H1}
\SEES{Clients, Central, droitDuTravail}
\VARIABLES
	\begin{description}
		\Item{ Ok }
		\Item{ Wait }
		\Item{ Frustrated }
		\Item{ Active }
		\Item{ OpClient }
		\Item{ NbPause }
	\end{description}
\INVARIANTS
	\begin{description}
		\nItem{ inv1 }{ Active \in  Operateurs \tfun  Etat }
		\nItem{ inv2 }{ OpClient \in  Operateurs \pinj  Wait }
		\nItem{ inv3 }{ dom(OpClient) = dom(Active \ranres  \{ reponse\} ) }
		\nItem{ inv4 }{ NbPause \in  Operateurs \tfun  0\upto pauseMax }
	\end{description}
\EVENTS
	\INITIALISATION
		\begin{description}
		\Begin
			\begin{description}
			\nItem{ act1 }{ Wait,Frustrated :=  \emptyset ,\emptyset  }
			\nItem{ act2 }{ Ok :=  CLIENTS }
			\nItem{ act3 }{ Active :=  Operateurs \cprod  \{ libre\}  }
			\nItem{ act4 }{ OpClient :=  Operateurs \cprod  \emptyset  }
			\nItem{ act5 }{ NbPause :=  Operateurs \cprod  \{ 0\}  }
			\end{description}
		\End
		\end{description}
	\EVT {Clac}
	\REF{Clac}
		\begin{description}
		\Any
			\begin{description}
			\Item{ c }
			\Item{ o }
			\end{description}
		\Where
			\begin{description}
			\nItem{ grd1 }{ c \in  Wait }
			\nItem{ grd2 }{ o \in  Operateurs }
			\nItem{ grd4 }{ Active(o) = reponse }
			\nItem{ grd3 }{ OpClient(o) = c }
			\end{description}
		\Then
			\begin{description}
			\nItem{ act1 }{ Wait,Frustrated :=  Wait\setminus \{ c\} ,Frustrated\bunion \{ c\}  }
			\nItem{ act2 }{ Active(o) :=  libre }
			\nItem{ act3 }{ OpClient :=  OpClient \setminus  \{ o \mapsto  c\}  }
			\end{description}
		\End
		\end{description}
	\EVT {Call}
		\begin{description}
			\Item{inherited}
		\end{description}
	\EVT {Panne}
		\begin{description}
			\Item{inherited}
		\end{description}
	\EVT {Fix}
	\REF{Fix}
		\begin{description}
		\Any
			\begin{description}
			\Item{ c }
			\Item{ o }
			\end{description}
		\Where
			\begin{description}
			\nItem{ grd1 }{ c \in  Wait }
			\nItem{ grd2 }{ o \in  Operateurs }
			\nItem{ grd4 }{ Active(o) = reponse }
			\nItem{ grd3 }{ OpClient(o) = c }
			\end{description}
		\Then
			\begin{description}
			\nItem{ act1 }{ Ok,Wait := Ok\bunion \{ c\} ,Wait\setminus \{ c\}  }
			\nItem{ act2 }{ Active(o) :=  libre }
			\nItem{ act3 }{ OpClient :=  OpClient \setminus  \{ o \mapsto  c\}  }
			\end{description}
		\End
		\end{description}
	\EVT {Break}
		\begin{description}
		\Any
			\begin{description}
			\Item{ o }
			\end{description}
		\Where
			\begin{description}
			\nItem{ grd1 }{ o \in  Operateurs }
			\nItem{ grd2 }{ Active(o) = libre }
			\nItem{ grd4 }{ NbPause(o) <  pauseMax }
			\nItem{ grd3 }{ card(Active \ranres  \{ libre\} ) > 
		\\\hspace*{1,2 cm}  card(Wait \setminus  ran(OpClient)) }
			\cmt{ Nombre d'opérateurs suffisant
pour traiter les clients en attente. }
			\end{description}
		\Then
			\begin{description}
			\nItem{ act1 }{ Active(o) :=  pause }
			\nItem{ act2 }{ NbPause(o) :=  NbPause(o)+1 }
			\end{description}
		\End
		\end{description}
	\EVT {UnBreak}
		\begin{description}
		\Any
			\begin{description}
			\Item{ o }
			\end{description}
		\Where
			\begin{description}
			\nItem{ grd1 }{ o \in  Operateurs }
			\nItem{ grd2 }{ Active(o) = pause }
			\end{description}
		\Then
			\begin{description}
			\nItem{ act1 }{ Active(o) :=  libre }
			\end{description}
		\End
		\end{description}
	\EVT {Answer}
		\begin{description}
		\Any
			\begin{description}
			\Item{ c }
			\Item{ o }
			\end{description}
		\Where
			\begin{description}
			\nItem{ grd3 }{ o \in  Operateurs }
			\nItem{ grd2 }{ c \in  Wait }
			\nItem{ grd1 }{ Active(o) = libre }
			\nItem{ grd4 }{ c \notin  ran(OpClient) }
			\end{description}
		\Then
			\begin{description}
			\nItem{ act1 }{ Active(o) :=  reponse }
			\nItem{ act2 }{ OpClient(o) :=  c }
			\end{description}
		\End
		\end{description}
\END
\end{description}
x

  \subsection{Machine H3}
  
   \subsubsection{Rôle de la machine H3}
   Le but de ce raffinement est de donner un ticket à un client lorsque
   celui-ci appelle la hotline. Nous gérons ce ticket grâce à une
   file. Nous ajoutons donc simplement une fonction partielle qui
   attribue un ticket à un client, le numéro global des tickets étant
   conservé dans une constante. Ainsi lorsqu'un client appelle, il se
   voit remettre un ticket qu'il gardera durant toute la communication
   que celle-ci se termine bien (\emph{Fix}) ou mal (\emph{Clac}).

   \thispagestyle{empty}
\begin{description}
\BTitle{H3}{11 Apr 2008}{04:17:18 PM}
\MACHINE{H3}
\REFINES{H2}
\SEES{Clients, Central, droitDuTravail}
\VARIABLES
	\begin{description}
		\Item{ Ok }
		\Item{ Wait }
		\Item{ Frustrated }
		\Item{ Active }
		\Item{ OpClient }
		\Item{ NbPause }
		\Item{ ClientTick }
		\Item{ NumTick }
	\end{description}
\INVARIANTS
	\begin{description}
		\nItem{ inv2 }{ ClientTick \in  CLIENTS \pinj  \nat }
		\nItem{ inv1 }{ NumTick \in  \nat }
		\cmt{ Numéro de ticket global. }
	\end{description}
\EVENTS
	\INITIALISATION
		\begin{description}
		\Begin
			\begin{description}
			\nItem{ act1 }{ Wait,Frustrated :=  \emptyset ,\emptyset  }
			\nItem{ act2 }{ Ok :=  CLIENTS }
			\nItem{ act3 }{ Active :=  Operateurs \cprod  \{ libre\}  }
			\nItem{ act4 }{ OpClient :=  Operateurs \cprod  \emptyset  }
			\nItem{ act5 }{ NbPause :=  Operateurs \cprod  \{ 0\}  }
			\nItem{ act6 }{ ClientTick :=  CLIENTS \cprod  \emptyset  }
			\nItem{ act7 }{ NumTick :=  0 }
			\end{description}
		\End
		\end{description}
	\EVT {Clac}
	\REF{Clac}
		\begin{description}
		\Any
			\begin{description}
			\Item{ c }
			\Item{ o }
			\end{description}
		\Where
			\begin{description}
			\nItem{ grd1 }{ c \in  Wait }
			\nItem{ grd2 }{ o \in  Operateurs }
			\nItem{ grd3 }{ Active(o) = reponse }
			\nItem{ grd4 }{ OpClient(o) = c }
			\nItem{ grd5 }{ c\in dom(ClientTick) }
			\end{description}
		\Then
			\begin{description}
			\nItem{ act1 }{ Wait,Frustrated :=  Wait\setminus \{ c\} ,Frustrated\bunion \{ c\}  }
			\nItem{ act2 }{ Active(o) :=  libre }
			\nItem{ act3 }{ OpClient :=  OpClient \setminus  \{ o \mapsto  c\}  }
			\nItem{ act4 }{ ClientTick :=  ClientTick \setminus  \{ c \mapsto  ClientTick(c)\}  }
			\end{description}
		\End
		\end{description}
	\EVT {Call}
	\REF{Call}
		\begin{description}
		\Any
			\begin{description}
			\Item{ c }
			\end{description}
		\Where
			\begin{description}
			\nItem{ grd1 }{ c\in Frustrated }
			\nItem{ grd2 }{ c\notin dom(ClientTick) }
			\nItem{ grd3 }{ NumTick\notin ran(ClientTick) }
			\end{description}
		\Then
			\begin{description}
			\nItem{ act1 }{ Wait,Frustrated:= Wait\bunion \{ c\} ,Frustrated\setminus \{ c\}  }
			\nItem{ act2 }{ ClientTick(c):= NumTick }
			\nItem{ act3 }{ NumTick:= NumTick+1 }
			\end{description}
		\End
		\end{description}
	\EVT {Panne}
		\begin{description}
			\Item{inherited}
		\end{description}
	\EVT {Fix}
	\REF{Fix}
		\begin{description}
		\Any
			\begin{description}
			\Item{ c }
			\Item{ o }
			\end{description}
		\Where
			\begin{description}
			\nItem{ grd1 }{ c \in  Wait }
			\nItem{ grd2 }{ o \in  Operateurs }
			\nItem{ grd3 }{ Active(o) = reponse }
			\nItem{ grd4 }{ OpClient(o) = c }
			\nItem{ grd5 }{ c\in dom(ClientTick) }
			\end{description}
		\Then
			\begin{description}
			\nItem{ act1 }{ Ok,Wait := Ok\bunion \{ c\} ,Wait\setminus \{ c\}  }
			\nItem{ act2 }{ Active(o) :=  libre }
			\nItem{ act3 }{ OpClient :=  OpClient \setminus  \{ o \mapsto  c\}  }
			\nItem{ act4 }{ ClientTick :=  ClientTick \setminus  \{ c \mapsto  ClientTick(c)\}  }
			\end{description}
		\End
		\end{description}
	\EVT {Break}
		\begin{description}
			\Item{inherited}
		\end{description}
	\EVT {UnBreak}
		\begin{description}
			\Item{inherited}
		\end{description}
	\EVT {Answer}
		\begin{description}
			\Item{inherited}
		\end{description}
\END
\end{description}


  \subsection{Machine H4}
  
   \subsubsection{Rôle de la machine H4}
   Le client se voit donc remettre un ticket quand il appel la
   hotline. Les opérateurs peuvent donc répondre en priorité au client
   qui a le plus attendu. Le système devient équitable. Pour ce faire,
   nous avons rajouté une garde à l'événement \emph{Answer} qui vérifie
   qu'il n'existe pas de client attendant qu'on lui réponde qui a un
   ticket inférieur à celui que l'on veut traiter (c'est-à-dire qui a
   attendu longtemps).

   \thispagestyle{empty}
\begin{description}
\BTitle{H4}{11 Apr 2008}{04:17:24 PM}
\MACHINE{H4}
\REFINES{H3}
\SEES{Clients, Central, droitDuTravail}
\VARIABLES
	\begin{description}
		\Item{ Ok }
		\Item{ Wait }
		\Item{ Frustrated }
		\Item{ Active }
		\Item{ OpClient }
		\Item{ NbPause }
		\Item{ ClientTick }
		\Item{ NumTick }
	\end{description}
\EVENTS
	\INITIALISATION
		\begin{description}
			\Item{inherited}
		\end{description}
	\EVT {Clac}
		\begin{description}
			\Item{inherited}
		\end{description}
	\EVT {Call}
		\begin{description}
			\Item{inherited}
		\end{description}
	\EVT {Panne}
		\begin{description}
			\Item{inherited}
		\end{description}
	\EVT {Fix}
		\begin{description}
			\Item{inherited}
		\end{description}
	\EVT {Break}
		\begin{description}
			\Item{inherited}
		\end{description}
	\EVT {UnBreak}
		\begin{description}
			\Item{inherited}
		\end{description}
	\EVT {Answer}
	\REF{Answer}
		\begin{description}
		\Any
			\begin{description}
			\Item{ c }
			\Item{ o }
			\end{description}
		\Where
			\begin{description}
			\nItem{ grd3 }{ o \in  Operateurs }
			\nItem{ grd2 }{ c \in  Wait }
			\nItem{ grd1 }{ Active(o) = libre }
			\nItem{ grd4 }{ c \notin  ran(OpClient) }
			\nItem{ grd6 }{ c \in  dom(ClientTick) }
			\nItem{ grd5 }{ \lnot (\exists cl\qdot (cl \notin  ran(OpClient) \land  
		\\\hspace*{1,2 cm}  (cl \in  Wait) \land  
		\\\hspace*{1,2 cm}  (cl \in  dom(ClientTick)) \land 
		\\\hspace*{1,2 cm}  (ClientTick(cl) \leq  ClientTick(c)))) }
			\cmt{ Le client qui a le ticket
le plus petit est prioritaire. }
			\end{description}
		\Then
			\begin{description}
			\nItem{ act1 }{ Active(o) :=  reponse }
			\nItem{ act2 }{ OpClient(o) :=  c }
			\end{description}
		\End
		\end{description}
\END
\end{description}


\end{document}
